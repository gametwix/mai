
\documentclass[pdf, unicode, 12pt, a4paper,oneside,fleqn]{article}
\usepackage{graphicx}
\graphicspath{{img/}}
\usepackage{log-style}
\begin{document}

\begin{titlepage}
    \begin{center}
        \bfseries

        {\Large Московский авиационный институт\\ (национальный исследовательский университет)}
        
        \vspace{48pt}
        
        {\large Факультет информационных технологий и прикладной математики}
        
        \vspace{36pt}
        
        {\large Кафедра вычислительной математики и~программирования}
        
        \vspace{48pt}
        
        Лабораторная работа \textnumero 2 по курсу \enquote{Операционные системы}

        \vspace{48pt}

        Управление процессами в ОС. Обеспечение обмена данных между процессами посредством каналов.
    \end{center}
    
    \vspace{150pt}
    
    \begin{flushright}
    \begin{tabular}{rl}
    Студент: & П.\,А. Мохляков \\
    Преподаватель: & Е.\,С. Миронов \\
    Группа: & М8О-208Б-19 \\
    Дата: & \\
    Оценка: & \\
    Подпись: & \\
    \end{tabular}
    \end{flushright}
    
    \vfill
    
    \begin{center}
    \bfseries
    Москва, \the\year
    \end{center}
\end{titlepage}
    
\pagebreak

\section{Постановка задачи}

Составить и отладить программу на языке Си, осуществляющую работу с процессами и
взаимодействие между ними в одной из двух операционных систем. В результате работы
программа (основной процесс) должен создать для решение задачи один или несколько
дочерних процессов. Взаимодействие между процессами осуществляется через системные
сигналы/события и/или каналы (pipe).

Необходимо обрабатывать системные ошибки, которые могут возникнуть в результате работы.

Родительский процесс создает дочерний процесс. Первой строчкой пользователь в консоль
родительского процесса пишет имя файла, которое будет передано при создании дочернего
процесса. Родительский и дочерний процесс должны быть представлены разными программами.

Родительский процесс передает команды пользователя через pipe1, который связан с
стандартным входным потоком дочернего процесса. Дочерний процесс принеобходимости
передает данные в родительский процесс через pipe2. Результаты своей работы дочерний
процесс пишет в созданный им файл. Допускается просто открыть файл и писать туда, не
перенаправляя стандартный поток вывода.

\section{Сведения о программе}

Программа написанна на Си в Unix подобной операционной системе на базе ядра Linux.
В программе создается дочерний процесс, в который перенаправляются данные из pipe.

Дочерний прочесс принимает строку чисел и находит их сумму, ответ записывая в файл. Имя файла задается пользователем

Родительский процесс считывает вводные данные у пользоветеля и пердет их дочернему процессу через pipe.

Программа завершает работу при оконцании ввода, то есть нажатии CTRL+D.

\section{Общий метод и алгоритм решения}



\end{document}
