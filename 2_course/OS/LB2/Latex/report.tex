
\documentclass[pdf, unicode, 12pt, a4paper,oneside,fleqn]{article}
\usepackage{graphicx}
\graphicspath{{img/}}
\usepackage{log-style}
\begin{document}

\begin{titlepage}
    \begin{center}
        \bfseries

        {\Large Московский авиационный институт\\ (национальный исследовательский университет)}
        
        \vspace{48pt}
        
        {\large Факультет информационных технологий и прикладной математики}
        
        \vspace{36pt}
        
        {\large Кафедра вычислительной математики и~программирования}
        
        \vspace{48pt}
        
        Лабораторная работа \textnumero 2 по курсу \enquote{Операционные системы}

        \vspace{48pt}

        Управление процессами в ОС. Обеспечение обмена данных между процессами посредством каналов.
    \end{center}
    
    \vspace{140pt}
    
    \begin{flushright}
    \begin{tabular}{rl}
    Студент: & П.\,А. Мохляков \\
    Преподаватель: & Е.\,С. Миронов \\
    Группа: & М8О-208Б-19 \\
    Вариант: & 1 \\
    Дата: & \\
    Оценка: & \\
    Подпись: & \\
    \end{tabular}
    \end{flushright}
    
    \vfill
    
    \begin{center}
    \bfseries
    Москва, \the\year
    \end{center}
\end{titlepage}
    
\pagebreak

\section{Постановка задачи}

Составить и отладить программу на языке Си, осуществляющую работу с процессами и
взаимодействие между ними в одной из двух операционных систем. В результате работы
программа (основной процесс) должен создать для решение задачи один или несколько
дочерних процессов. Взаимодействие между процессами осуществляется через системные
сигналы/события и/или каналы (pipe).

Необходимо обрабатывать системные ошибки, которые могут возникнуть в результате работы.

Родительский процесс создает дочерний процесс. Первой строчкой пользователь в консоль
родительского процесса пишет имя файла, которое будет передано при создании дочернего
процесса. Родительский и дочерний процесс должны быть представлены разными программами.

Родительский процесс передает команды пользователя через pipe1, который связан с
стандартным входным потоком дочернего процесса. Дочерний процесс принеобходимости
передает данные в родительский процесс через pipe2. Результаты своей работы дочерний
процесс пишет в созданный им файл. Допускается просто открыть файл и писать туда, не
перенаправляя стандартный поток вывода.

\section{Сведения о программе}

Программа написанна на Си в Unix подобной операционной системе на базе ядра Linux.
В программе создается дочерний процесс, в который перенаправляются данные из pipe.

Дочерний прочесс принимает строку чисел и находит их сумму, ответ записывая в файл. Имя файла задается пользователем

Родительский процесс считывает вводные данные у пользоветеля и пердет их дочернему процессу через pipe.

Программа завершает работу при оконцании ввода, то есть нажатии CTRL+D.

\section{Общий метод и алгоритм решения}

При запуске программы прользоваательль может ввести имя файла, который создаст дочерняя программа.
Считывание происходит посредством \textbf{getline()}.

После запуска создается pipe, два файловых дескриптора которого, записываюся в массив fd из двех элементов.

После этого создается дочерний процесс с помощью \textbf{fork()}. В нем дескриптор потока ввода заменяется на поток вывода из pipe.
Таким образом, когда родитель запишет что-то в pipe ребенок сможет это считать, как будто ввод происходит из консоли. После замены 
дескрипторов вызывается дочерняя прогграмма с помощью \textbf{execl()}. В нее имя файла передается как параметр при запуске. 
Дальее пака не стретим конец ввода мы считываем число и символ за ним. Если этот символ пробел, то считанное число просто прибавляется к сумме, если 
символ является символом конца строки, сумма записывается в файл, а сумма обнуляется.

Тем временем родитель посимвольно считывает данные от пользователя и записывает их в файл.
При нажатии CTRL+D пользователь сигнализирует о конце ввода. Родительский процесс завершает работу,
а вместе с ним и дочерний.

\section{Листинг программы}

{\large\textbf{main.c}}

\begin{lstlisting}[language=C]
#include "unistd.h"
#include "string.h"
#include "stdio.h"
int main(){
    int fd[2];
    if(pipe(fd) < 0){
        printf("Error pipe create\n");
        return -1;
    }
    char *filename = NULL;
    size_t sizename = 0;
    getline(&filename,&sizename,stdin);
    filename[strlen(filename)-1] = '\0';
    int id = fork();
    if(id == -1){
        printf("Error fork\n");
        return -1;
    }else if(id == 0){
        close(fd[1]);
        dup2(fd[0],0);
        execl("./child","child",filename,(char*) NULL);
    } else {
        close(fd[0]);
        char ch;
        while(scanf("%c",&ch) != EOF){
            write(fd[1],&ch,sizeof(ch));  
        }
        close(fd[1]);
    }
    return 0;
}
\end{lstlisting}

{\large\textbf{sum.c}}

\begin{lstlisting}[language=C]
#include "stdio.h"
#include "string.h"

int main(int argc, char **argv){
    if(argc != 2){
        return -1;
    }
    long long sum = 0;
    int num;
    char ch;
    FILE *file;
    file = fopen(argv[1], "w");
    while(scanf("%d%c",&num,&ch) != EOF){
        sum += num;
        if(ch == '\n'){
            fprintf(file,"%lld\n",sum);
            sum = 0;
        }
    }
    fclose(file);
    return 0;
}
\end{lstlisting}

\section{Демонстрация работы программы}

\begin{alltt}
pavel@DESKTOP-K5KMLPV:~/Project/mai/2_course/OS/LB2$ gcc main.c -o parent
pavel@DESKTOP-K5KMLPV:~/Project/mai/2_course/OS/LB2$ gcc sum.c -o child
pavel@DESKTOP-K5KMLPV:~/Project/mai/2_course/OS/LB2$ ./parent
test
1 2 3 4 5
0 0 0
12 45 34 54
24 -5
pavel@DESKTOP-K5KMLPV:~/Project/mai/2_course/OS/LB2$ cat test
15
0
145
19
pavel@DESKTOP-K5KMLPV:~/Project/mai/2_course/OS/LB2$ strace -f -e 
trace="read,write,dup2,pipe" -o log.txt ./parent
test2
1 2 3
0 0
2 -1
pavel@DESKTOP-K5KMLPV:~/Project/mai/2_course/OS/LB2$ cat log.txt
2452  read(3, "\textbackslash177ELF\textbackslash\textbackslash2\textbackslash1\textbackslash1\textbackslash3\textbackslash0\textbackslash0\textbackslash0\textbackslash0\textbackslash0\textbackslash0\textbackslash0\textbackslash0\textbackslash3\textbackslash0>\textbackslash0\textbackslash1\textbackslash0\textbackslash0\textbackslash0\textbackslash360q
\textbackslash2\textbackslash0\textbackslash0\textbackslash0\textbackslash0\textbackslash0"..., 832) = 832
2452  pipe([3, 4])                      = 0
2452  read(0, "test2\textbackslash n", 1024)          = 6
2452  read(0,  <unfinished ...>
2454  dup2(3, 0)                        = 0
2454  read(4, "\textbackslash177ELF\textbackslash2\textbackslash1\textbackslash1\textbackslash3\textbackslash0\textbackslash0\textbackslash0\textbackslash0\textbackslash0\textbackslash0\textbackslash0\textbackslash0\textbackslash3\textbackslash0>\textbackslash0\textbackslash1\textbackslash0\textbackslash0\textbackslash0\textbackslash360q
\textbackslash2\textbackslash0\textbackslash0\textbackslash0\textbackslash0\textbackslash0"..., 832) = 832
2454  read(0,  <unfinished ...>
2452  <... read resumed>"1 2 3\textbackslash{n}", 1024) = 6
2452  write(4, "1", 1)                  = 1
2454  <... read resumed>"1", 4096)      = 1
2452  write(4, " ", 1 <unfinished ...>
2454  read(0,  <unfinished ...>
2452  <... write resumed>)              = 1
2454  <... read resumed>" ", 4096)      = 1
2452  write(4, "2", 1 <unfinished ...>
2454  read(0,  <unfinished ...>
2452  <... write resumed>)              = 1
2454  <... read resumed>"2", 4096)      = 1
2452  write(4, " ", 1 <unfinished ...>
2454  read(0,  <unfinished ...>
2452  <... write resumed>)              = 1
2454  <... read resumed>" ", 4096)      = 1
2452  write(4, "3", 1 <unfinished ...>
2454  read(0,  <unfinished ...>
2452  <... write resumed>)              = 1
2454  <... read resumed>"3", 4096)      = 1
2452  write(4, "\textbackslash{n}", 1 <unfinished ...>
2454  read(0,  <unfinished ...>
2452  <... write resumed>)              = 1
2454  <... read resumed>"\textbackslash{n}", 4096)     = 1
2452  read(0,  <unfinished ...>
2454  read(0,  <unfinished ...>
2452  <... read resumed>"0 0\textbackslash{n}", 1024)  = 4
2452  write(4, "0", 1)                  = 1
2454  <... read resumed>"0", 4096)      = 1
2452  write(4, " ", 1 <unfinished ...>
2454  read(0,  <unfinished ...>
2452  <... write resumed>)              = 1
2454  <... read resumed>" ", 4096)      = 1
2452  write(4, "0", 1)                  = 1
2454  read(0,  <unfinished ...>
2452  write(4, "\textbackslash{n}", 1)                 = 1
2454  <... read resumed>"0\textbackslash{n}", 4096)    = 2
2452  read(0,  <unfinished ...>
2454  read(0,  <unfinished ...>
2452  <... read resumed>"2 -1\textbackslash{n}", 1024) = 5
2452  write(4, "2", 1)                  = 1
2454  <... read resumed>"2", 4096)      = 1
2452  write(4, " ", 1 <unfinished ...>
2454  read(0,  <unfinished ...>
2452  <... write resumed>)              = 1
2454  <... read resumed>" ", 4096)      = 1
2452  write(4, "-", 1 <unfinished ...>
2454  read(0,  <unfinished ...>
2452  <... write resumed>)              = 1
2454  <... read resumed>"-", 4096)      = 1
2452  write(4, "1", 1 <unfinished ...>
2454  read(0,  <unfinished ...>
2452  <... write resumed>)              = 1
2454  <... read resumed>"1", 4096)      = 1
2452  write(4, "\textbackslash{n}", 1 <unfinished ...>
2454  read(0,  <unfinished ...>
2452  <... write resumed>)              = 1
2454  <... read resumed>"\textbackslash{n}", 4096)     = 1
2452  read(0,  <unfinished ...>
2454  read(0,  <unfinished ...>
2452  <... read resumed>"", 1024)       = 0
2454  <... read resumed>"", 4096)       = 0
2454  write(4, "6\textbackslash{n}0\textbackslash{n}1\textbackslash{n}", 6)          = 6
2452  +++ exited with 0 +++
2454  +++ exited with 0 +++
\end{alltt}

\pagebreak

\section{Вывод}

Одна из основных задач операционной системы - это управление процессами.
В большинстве случаев она сама создает процессы для себя и при запуске других программ.
Тем не менее бывают случаи, когда необходимо создавать процессы вручную.

В языке Си есть функционал, который позволит нам внутри нашей программы создать
дополнительный, дочерний процесс. Этот процесс будет работать параллельно с родительским.

Для этого в языке Си на Unix-подобных ОС используется библеотека unistd.h.
Эта библеотека позволяет совершать системные вызовы, которые связаны с 
вводом/выводом, управлением файлами, каталогами и работой с процессами и запуском программ.
Для создания дочерних процессов используется функция fork. При этом с помощью ветвлений 
в коде можно отделить код родителя от ребенка. У ребенка при этом можно заменить программу,
испрользуя для этого функцию exec, а обеспечить связь с помощью pipe.

Подобный функционал есть во многих языках программирования, так как большинство современных программ состаят более, чем из одного процесса.




\end{document}


