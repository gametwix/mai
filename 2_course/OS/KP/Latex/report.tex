
\documentclass[pdf, unicode, 12pt, a4paper,oneside,fleqn]{article}
\usepackage{graphicx}
\graphicspath{{img/}}
\usepackage{log-style}
\begin{document}

\begin{titlepage}
    \begin{center}
        \bfseries

        {\Large Московский авиационный институт\\ (национальный исследовательский университет)}
        
        \vspace{48pt}
        
        {\large Факультет информационных технологий и прикладной математики}
        
        \vspace{36pt}
        
        {\large Кафедра вычислительной математики и~программирования}
        
        \vspace{48pt}
        
        Курсовой проект по курсу <<Операционные системы>>

    \end{center}
    
    \vspace{140pt}
    
    \begin{flushright}
    \begin{tabular}{rl}
    Студент: & П.\,А. Мохляков \\
    Преподаватель: & Е.\,С. Миронов \\
    Группа: & М8О-208Б-19 \\
    Дата: & \\
    Оценка: & \\
    Подпись: & \\
    \end{tabular}
    \end{flushright}
    
    \vfill
    
    \begin{center}
    \bfseries
    Москва, \the\year
    \end{center}
\end{titlepage}
    
\pagebreak

\section{Постановка задачи}

Необходимо написать 3-и программы. Далее будем обозначать эти программы A, B, C.

Программа A принимает из стандартного потока ввода строки, а далее их отправляет программе С. Отправка строк должна производится построчно.Программа C печатает в стандартый вывод, полученную строку от программы A. После получения программа C отправляет программе А сообщение о том, что строка получена. До тех пор пока программа А не примет«сообщение о получение строки» от программы С, она не может отправялять следующую строку программе С.
Программа B пишет в стандартный вывод количество отправленных символов программой А и количество принятыхсимволов программой С. Данную информацию программа B получает от программ A и C соответственно

\section{Сведения о программе}

Программы написаны на языке C++ для Unix подобной операционной системы на базе ядра Linux.
Для связи между программ используются сокеты с помощью библиотеки ZeroMQ.

Для работы запустите все три программы. Программа А считывает строки у пользователя пока пользователь не завершит ввод, после чего передает строки программе С. Та в свою очередь выводит их на экран. Обе программы отправляют длины отправленных
и полученных строк программе B. Она выводит их на экран, по этим данным можно проверить не потерялись ли данные при отправке.

\section{Общий метод и алгоритм решения}
Программа А подключается к программе B и С. После чего она считывает строки от пользователя и записывает их в вектор. При нажатии CTRL+D пользователь сигнализирует о конце ввода. 
Далее мы ждем подключение от программы С и начинаем передавать строки. В цикле передаем программе С троку и ждем сообщение об удачной передаче строки.
После чего мы передаем программе B длину строки и также ждем ответа об удачной передаче. При завершеении работы программа A передает сообщения программам B и C,
чтобы они также завершили работу.

Программа С отправляет сообщение об удачном подключении к программе А, после чего попадает в бесконечный цикл.
Там мы считываем сообщение от программы A, отправяем сообщение об учасном принятии сообщения и выводим сообщение на экран. Далее мы также отправляем 
программе сообщение с длиной принятой строки и ждем от нее ответа об удачной передачи и повторяем цикл. Если мы приняли сообщение о завершении работы программы А, то мы входим из цикла и сами завершаем работу.

Программа B принимает длины строк от программы А и С и выводит их на экран, после чего повторяет цикл. При пололучении сообщения о завершении работы программы А, входит из цикла и сами завершает работу. 


\section{Листинг программы}

{\large\textbf{A.cpp}}

\begin{lstlisting}[language=C++]
#include <iostream>
#include <string>
#include <vector>
#include <zmq.hpp>

#define ADDRESS_C "tcp://127.0.0.1:5555"
#define ADDRESS_B "tcp://127.0.0.1:5556"

int main(){
    zmq::context_t context(1);
    zmq::socket_t to_c(context,ZMQ_REP);
    zmq::socket_t to_b(context,ZMQ_REQ);

    to_c.bind(ADDRESS_C);
    to_b.connect(ADDRESS_B);

    std::string str;
    std::vector<std::string> all_strings;
    while(std::getline(std::cin,str)){
        all_strings.push_back(str);
    }

    std::string ans;
    zmq::message_t message;
    to_c.recv(message);
    ans = std::string(static_cast<char*>(message.data()), message.size());
    if(ans != "Connect")
        return 1;

    for(auto& string: all_strings){
        message = zmq::message_t(string.size());
        memcpy(message.data(), string.c_str(), string.size());
        to_c.send(message);
        to_c.recv(message);
        ans = std::string(static_cast<char*>(message.data()), message.size());
        if(ans != "Success")
            break;
        std::string size = std::to_string(string.size());
        message = zmq::message_t(size.size());
        memcpy(message.data(), size.c_str(), size.size());
        to_b.send(message);
        to_b.recv(message);
        ans = std::string(static_cast<char*>(message.data()), message.size());
        if(ans != "OK")
            break;
    }
    ans = "$$$";
    message = zmq::message_t(ans.size());
    memcpy(message.data(), ans.c_str(), ans.size());
    to_c.send(message);

    ans = "-1";
    message = zmq::message_t(ans.size());
    memcpy(message.data(), ans.c_str(), ans.size());
    to_b.send(message);

    to_c.unbind(ADDRESS_C);
    to_b.disconnect(ADDRESS_B);

    return 0;
}
\end{lstlisting}

{\large\textbf{C.cpp}}

\begin{lstlisting}[language=C++]
#include <iostream>
#include <string>
#include <zmq.hpp>

#define ADDRESS_A "tcp://127.0.0.1:5556"
#define ADDRESS_C "tcp://127.0.0.1:5557"

int main(){
    zmq::context_t context(1);
    zmq::socket_t to_a(context,ZMQ_REP);
    zmq::socket_t to_c(context,ZMQ_REP);
    std::string my_ans = "OK";
    to_a.bind(ADDRESS_A);
    to_c.bind(ADDRESS_C);
    while(true){
        zmq::message_t message_a;
        to_a.recv(message_a);
        std::string ans_a = std::string(static_cast<char*>(message_a.data()), message_a.size());
        int size_str_a = std::stoi(ans_a);
        if(size_str_a == -1)
            break;
        message_a = zmq::message_t(my_ans.size());
        memcpy(message_a.data(), my_ans.c_str(), my_ans.size());
        to_a.send(message_a);

        zmq::message_t message_c;
        to_c.recv(message_a);
        std::string ans_c = std::string(static_cast<char*>(message_a.data()), message_a.size());
        int size_str_c = std::stoi(ans_c);
        message_c = zmq::message_t(my_ans.size());
        memcpy(message_c.data(), my_ans.c_str(), my_ans.size());
        to_c.send(message_c);

        std::cout << "Size str send A: " << size_str_a << "\tSize str send C: " << size_str_c << std::endl;

    }
    to_a.unbind(ADDRESS_A);
    to_c.unbind(ADDRESS_C);
    return 0;
}
\end{lstlisting}

{\large\textbf{B.cpp}}

\begin{lstlisting}[language=C++]
#include <iostream>
#include <string>
#include <zmq.hpp>

#define ADDRESS_A "tcp://127.0.0.1:5556"
#define ADDRESS_C "tcp://127.0.0.1:5557"

int main(){
    zmq::context_t context(1);
    zmq::socket_t to_a(context,ZMQ_REP);
    zmq::socket_t to_c(context,ZMQ_REP);
    std::string my_ans = "OK";
    to_a.bind(ADDRESS_A);
    to_c.bind(ADDRESS_C);
    while(true){
        zmq::message_t message_a;
        to_a.recv(message_a);
        std::string ans_a = std::string(static_cast<char*>(message_a.data()), message_a.size());
        int size_str_a = std::stoi(ans_a);
        if(size_str_a == -1)
            break;
        message_a = zmq::message_t(my_ans.size());
        memcpy(message_a.data(), my_ans.c_str(), my_ans.size());
        to_a.send(message_a);

        zmq::message_t message_c;
        to_c.recv(message_a);
        std::string ans_c = std::string(static_cast<char*>(message_a.data()), message_a.size());
        int size_str_c = std::stoi(ans_c);
        message_c = zmq::message_t(my_ans.size());
        memcpy(message_c.data(), my_ans.c_str(), my_ans.size());
        to_c.send(message_c);

        std::cout << "Size str send A: " << size_str_a << "\tSize str send C: " << size_str_c << std::endl;

    }
    to_a.unbind(ADDRESS_A);
    to_c.unbind(ADDRESS_C);
    return 0;
}
\end{lstlisting}

\section{Демонстрация работы программ}

{\large\textbf{Программа А}}
\begin{alltt}
pavel@DESKTOP-K5KMLPV:~/Project/mai/2_course/OS/KP$ ./A
hello world
check
test
I'm working
\end{alltt}

{\large\textbf{Программа С}}
\begin{alltt}
pavel@DESKTOP-K5KMLPV:~/Project/mai/2_course/OS/KP$ ./C
hello world
check
test
I'm working
\end{alltt}

{\large\textbf{Программа B}}
\begin{alltt}
pavel@DESKTOP-K5KMLPV:~/Project/mai/2_course/OS/KP$ ./B
Size str send A: 11     Size str send C: 11
Size str send A: 5      Size str send C: 5
Size str send A: 4      Size str send C: 4
Size str send A: 11     Size str send C: 11
\end{alltt}

\pagebreak

\section{Вывод}

Данная крусовая работа основывается на знаниях полученных в ходе изучения курса. По итогу мы получили нейколько программ, которые взаимодействуют друг с другом с помощью сокетов.
Задача курсового проекта не сложна в реализации, но ее реализация обощает и закрепляет полученные в курсе знания.

\end{document}


