
\documentclass[pdf, unicode, 12pt, a4paper,oneside,fleqn]{article}
\usepackage{graphicx}
\graphicspath{{img/}}
\usepackage{log-style}
\begin{document}

\begin{titlepage}
    \begin{center}
        \bfseries

        {\Large Московский авиационный институт\\ (национальный исследовательский университет)}
        
        \vspace{48pt}
        
        {\large Факультет информационных технологий и прикладной математики}
        
        \vspace{36pt}
        
        {\large Кафедра вычислительной математики и~программирования}
        
        \vspace{48pt}
        
        Лабораторная работа \textnumero 3 по курсу \enquote{Операционные системы}

        \vspace{48pt}

        Управление потоками в ОС. Обеспечение синхронизации между потоками
    \end{center}
    
    \vspace{140pt}
    
    \begin{flushright}
    \begin{tabular}{rl}
    Студент: & П.\,А. Мохляков \\
    Преподаватель: & Е. \,С. Миронов \\
    Группа: & М8О-208Б-19 \\
    Вариант: & 1 \\
    Дата: & \\
    Оценка: & \\
    Подпись: & \\
    \end{tabular}
    \end{flushright}
    
    \vfill
    
    \begin{center}
    \bfseries
    Москва, \the\year
    \end{center}
\end{titlepage}
    
\pagebreak

\section{Постановка задачи}

Составить программу на языке Си, обрабатывающую данные в многопоточном
режиме. При обработки использовать стандартные средства создания потоков
операционной системы (Windows/Unix). Ограничение потоков должно быть задано 
ключом запуска вашей программы.

Так же необходимо уметь продемонстрировать количество потоков, используемое 
вашей программой с помощью стандартных средств операционной системы.

В отчете привести исследование зависимости ускорения и эффективности 
алгоритма от входящих данных и количества потоков. Получившиеся результаты 
необходимо объяснить.

Отсортировать массив целых чисел при помощи битонической сортировки.

\section{Сведения о программе}

Программа написанна на Си в Unix подобной операционной системе на базе ядра Linux.
Для компиляции требуется ключ -lpthread, для запуска программы нелбходимо указать
в качестве аргумента количество потоков, которые максимально могут быть использованы.

Программа сортирует массив с помощью битонной сортировки.
Сначала пользователь должен ввести колличество элементов массива.
Далее должен передать все элементы.

Существует три глобальных переменных, которые являются мьютексом и 2 переменнами колличества потоков,
используемые и максимальные. Их использование между потоками регулируется мьютексом.
Сортируемый массив передается как указатель. И для него не нужен мьютекс, так как 
каждый поток работает на своем участке, не перекрываемом другими.

\section{Общий метод и алгоритм решения}

Параментром запуска программы мы указваем максимальное количество использумых потоков.

Далее мы указываем длину массива, и создаем массив дополив длину до ближайшего числа $2^n$.
Так как это необходимо для работы алгоритма сортировки. Далее мы считываем 
данные, а оставшийся участок массива заполняем максимально возможными элементами,
чтобы после сортировки они оказалить в конце массива, и мы могли также легко их удалить.

Далее вызывается функция сортировки. Сначала мы рекурсивно разбиваем 
массив по полам, задавая целевое направление сначала вниз, а потом вверх, чтобы получить
битонную последовательность. При этом пока у нас есть свободные потоки мы выдиляем под вторую
половину отдельный поток, а когда они закончатся начинаем работать в однопотоке, на выделенном участке.

Когда мы достигаем массива из одного элемента его можно считать отсортированным, как по возрастанию так
и по убыванию. Поэтому два соседних элемента можно считать битонной последовательностью, которую можно собрать в одну
возрастающую или убывающую.

Начинаем объядинять битонные последовательности. Находим расстояние между двумя
элиментами, которые будем сравнивать, оно равно половине участка массива.
Далее мы проходим по половине участка и сравниваем каждый элемент с элементом через заданное расстояние.
При необходимости меняем их местами. Далее разбиваем данный участок на 2 и повторяем слияние, и так пока его длина не станет равна 1.

Таким образом мы получаем битоническую последовательность большего размера, к которой можно опять применить слияние.
Повторяем эту процедуру до оканчания сортироки. При этом когда мы будем делать слияние для 2 участков, созданных
разными потоками, слияние мы опять разбиваем на несколько потоков, пока это возможно.

При всем этом если один поток подготовил последовательность, а второй еще нет, то первый его ждет, из чего
следует вывод, что увеличение производительности будет только при увеличении количества потоков
до следующей степени двойки.

\section{Листинг программы}

{\large\textbf{main.c}}

\begin{lstlisting}[language=C]
#include "stdio.h"
#include "stdlib.h"
#include "bitonic.h"
#include <sys/time.h>

#define MAXINT 2147483647

int SizeStep(int Num){
    int i = 1;
    while(i < Num)
        i *= 2;
    return i;
}


int main(int argc, char *argv[]){
    int threads = 1;

    if(argc == 2){
        threads = atoi(argv[1]);
    }

    int input_size;
    scanf("%d",&input_size);

    int size_array = SizeStep(input_size); 
    int *array = malloc(sizeof(int)*size_array);
    
    for(int i = 0; i < input_size; ++i)
        scanf("%d",array+i);
    for(int i = input_size; i < size_array; ++i)
        array[i] = MAXINT;

    #ifdef time
    struct timeval startwtime, endwtime;
    gettimeofday(&startwtime, NULL);
    #endif

    bitonicsort(array, size_array, threads);

    #ifdef time
    gettimeofday(&endwtime, NULL);
    double time = (double)((endwtime.tv_usec - startwtime.tv_usec)/1.0e6 + endwtime.tv_sec - startwtime.tv_sec);
    printf("%f\n", time);
    #endif


    #ifndef time
    for(int i=0;i<input_size;++i){
        printf("%d\n",array[i]);
    }
    #endif

    free(array);
    return 0;
}
\end{lstlisting}

{\large\textbf{bitonuc.h}}

\begin{lstlisting}[language=C]
#pragma once

#include "pthread.h"

#define UP 1
#define DOWN 0

typedef struct ArgsBitonic{
    int *array;
    int size;
    int start;
    int dir;
}ArgsBitonic;

void InitArgs(ArgsBitonic *args, int *array, int size, int start, int dir);
void Comparator(int *array, int i, int j, int dir);
void BitonicMergeSinglThread(ArgsBitonic *args);
void BitonicSortSinglThread(ArgsBitonic *args);
void BitonicMergeMultiThreads(ArgsBitonic *args);
void BitonicSortMultiThreads(ArgsBitonic *args);
void bitonicsort(int *array, int size, int threads);
\end{lstlisting}

{\large\textbf{bitonuc.c}}

\begin{lstlisting}[language=C]
#include "pthread.h"
#include "bitonic.h"
#include "stdio.h"

#define UP 1
#define DOWN 0

pthread_mutex_t lock;
size_t max_threads = 1;
size_t use_threads = 1;

void InitArgs(ArgsBitonic *args, int *array, int size, int start, int dir){
    args->array = array;
    args->size  = size;
    args->start = start;
    args->dir   = dir;
}


void Comparator(int *array, int i, int j, int dir){
    if(dir == (array[i] > array[j])){
        int temp = array[i];
        array[i] = array[j];
        array[j] = temp;
    }
}

void BitonicMergeSinglThread(ArgsBitonic *args){
    if(args->size > 1){
        int nextsize = args->size / 2;
        for(int i = args->start; i < nextsize + args->start; ++i){
                Comparator(args->array, i, i + nextsize, args->dir);
        }

        ArgsBitonic args1;
        ArgsBitonic args2;
        InitArgs(&args1, args->array, nextsize, args->start, args->dir);
        InitArgs(&args2, args->array, nextsize, args->start + nextsize, args->dir);
        
        BitonicMergeSinglThread(&args1);
        BitonicMergeSinglThread(&args2);
    }
}

void BitonicSortSinglThread(ArgsBitonic *args){
    if(args->size > 1){
        int nextsize = args->size / 2;

        ArgsBitonic args1;
        ArgsBitonic args2;
        InitArgs(&args1, args->array, nextsize, args->start, DOWN);
        InitArgs(&args2, args->array, nextsize, args->start + nextsize, UP);

        BitonicSortSinglThread(&args1);
        BitonicSortSinglThread(&args2);
        BitonicMergeSinglThread(args);
    }
}

void BitonicMergeMultiThreads(ArgsBitonic *args){
    if(args->size > 1){        
        int nextsize = args->size / 2;
        int isParal = 0;
        pthread_t tid;

        for(int i = args->start; i < nextsize + args->start; ++i){
                Comparator(args->array, i, i + nextsize, args->dir);
        }

        ArgsBitonic args1;
        ArgsBitonic args2;
        InitArgs(&args1, args->array, nextsize, args->start, args->dir);
        InitArgs(&args2, args->array, nextsize, args->start + nextsize, args->dir);

        pthread_mutex_lock(&lock);
        if(use_threads < max_threads){ 
            ++use_threads;
            pthread_mutex_unlock(&lock);
            isParal = 1;
            pthread_create(&tid, NULL,(void*) &BitonicMergeMultiThreads, &args1);
            BitonicMergeMultiThreads(&args2);
        } else {
            pthread_mutex_unlock(&lock);
            BitonicMergeSinglThread(&args1);
            BitonicMergeSinglThread(&args2);
        }

        if(isParal){
            pthread_join(tid, NULL);
            pthread_mutex_lock(&lock);
            --use_threads;
            pthread_mutex_unlock(&lock);
        }
    }
}
    
void BitonicSortMultiThreads(ArgsBitonic *args){
    if(args->size > 1 ){
        int nextsize = args->size / 2;
        int isParal = 0;
        pthread_t tid;

        ArgsBitonic args1;
        ArgsBitonic args2;
        InitArgs(&args1, args->array, nextsize, args->start, DOWN);
        InitArgs(&args2, args->array, nextsize, args->start + nextsize, UP);

        pthread_mutex_lock(&lock);
        if(use_threads < max_threads){
            ++use_threads;
            pthread_mutex_unlock(&lock);
            isParal = 1;
            pthread_create(&tid, NULL,(void*) &BitonicSortMultiThreads, &args1);
            BitonicSortMultiThreads(&args2);
        } else {
            pthread_mutex_unlock(&lock);
            BitonicSortSinglThread(&args1);
            BitonicSortSinglThread(&args2);
        }

        if(isParal){
            pthread_join(tid, NULL);
            pthread_mutex_lock(&lock);
            --use_threads;
            pthread_mutex_unlock(&lock);
        }
        BitonicMergeMultiThreads(args);
    }
}

void bitonicsort(int *array, int size, int threads){
    pthread_mutex_init(&lock, NULL);

    ArgsBitonic args;
    InitArgs(&args,array,size,0,UP);

    if(threads > 1)
        max_threads = threads;

    BitonicSortMultiThreads(&args);
    
    pthread_mutex_destroy(&lock);
}
\end{lstlisting}

\section{Демонстрация работы программы}

\begin{alltt}
pavel@DESKTOP-K5KMLPV:~/Project/mai/2_course/OS/LB3$ make clean
rm -r *.o main
pavel@DESKTOP-K5KMLPV:~/Project/mai/2_course/OS/LB3$
pavel@DESKTOP-K5KMLPV:~/Project/mai/2_course/OS/LB3$ make
gcc -c -Wall src/main.c
gcc -c -Wall src/bitonic.c
gcc main.o bitonic.o -pthread -o main
pavel@DESKTOP-K5KMLPV:~/Project/mai/2_course/OS/LB3$ ./main 1
5
5 4 3 2 1
1
2
3
4
5
pavel@DESKTOP-K5KMLPV:~/Project/mai/2_course/OS/LB3$ ./main 2
8
23 54 2 -12 6 2 13423354 -1231
-1231
-12
2
2
6
23
54
13423354
pavel@DESKTOP-K5KMLPV:~/Project/mai/2_course/OS/LB3$ ./main 4
4
2 5 0 -10
-10
0
2
5
pavel@DESKTOP-K5KMLPV:~/Project/mai/2_course/OS/LB3$ cat test | ./main 1
28.916599
pavel@DESKTOP-K5KMLPV:~/Project/mai/2_course/OS/LB3$ cat test | ./main 2
15.838372
pavel@DESKTOP-K5KMLPV:~/Project/mai/2_course/OS/LB3$ cat test | ./main 3
15.663935
pavel@DESKTOP-K5KMLPV:~/Project/mai/2_course/OS/LB3$ cat test | ./main 4
9.641949
pavel@DESKTOP-K5KMLPV:~/Project/mai/2_course/OS/LB3$ cat test | ./main 5
9.792694
pavel@DESKTOP-K5KMLPV:~/Project/mai/2_course/OS/LB3$ cat test | ./main 6
10.508794
pavel@DESKTOP-K5KMLPV:~/Project/mai/2_course/OS/LB3$ cat test | ./main 7
10.540718
pavel@DESKTOP-K5KMLPV:~/Project/mai/2_course/OS/LB3$ cat test | ./main 8
11.463141
pavel@DESKTOP-K5KMLPV:~/Project/mai/2_course/OS/LB3$ strace -f -e 
trace="%process,write" -o log.txt ./main 4
8
23 55 -36 7 0 254 -454 3
-454
-36
0
3
7
23
55
254
pavel@DESKTOP-K5KMLPV:~/Project/mai/2_course/OS/LB3$ cat log.txt
1273  execve("./main", ["./main", "4"], 0x7ffda4153150 /* 29 vars */) = 0
1273  arch_prctl(0x3001 /* ARCH_??? */, 0x7ffc25e05a30) = -1 EINVAL 
(Invalid argument)
1273  arch_prctl(ARCH_SET_FS, 0x7f2ea1d04740) = 0
1273  clone(child_stack=0x7f2ea1d02fb0, flags=CLONE_VM|CLONE_FS|
CLONE_FILES|CLONE_SIGHAND|CLONE_THREAD|CLONE_SYSVSEM|CLONE_SETTLS
|CLONE_PARENT_SETTID|CLONE_CHILD_CLEARTID, parent_tid=[1274],
tls=0x7f2ea1d03700, child_tidptr=0x7f2ea1d039d0) = 1274
1273  clone(child_stack=0x7f2ea1501fb0, flags=CLONE_VM|CLONE_FS|
CLONE_FILES|CLONE_SIGHAND|CLONE_THREAD|CLONE_SYSVSEM|CLONE_SETTLS|
CLONE_PARENT_SETTID|CLONE_CHILD_CLEARTID, parent_tid=[1275], 
tls=0x7f2ea1502700, child_tidptr=0x7f2ea15029d0) = 1275
1275  exit(0)                           = ?
1274  clone(child_stack=0x7f2ea0d00fb0, flags=CLONE_VM|CLONE_FS|
CLONE_FILES|CLONE_SIGHAND|CLONE_THREAD|CLONE_SYSVSEM|CLONE_SETTLS|
CLONE_PARENT_SETTID|CLONE_CHILD_CLEARTID <unfinished ...>
1275  +++ exited with 0 +++
1273  clone(child_stack=0x7f2ea1501fb0, flags=CLONE_VM|CLONE_FS|
CLONE_FILES|CLONE_SIGHAND|CLONE_THREAD|CLONE_SYSVSEM|CLONE_SETTLS|
CLONE_PARENT_SETTID|CLONE_CHILD_CLEARTID <unfinished ...>
1274  <... clone resumed>, parent_tid=[1276], tls=0x7f2ea0d01700, 
child_tidptr=0x7f2ea0d019d0) = 1276
1273  <... clone resumed>, parent_tid=[1277], tls=0x7f2ea1502700, 
child_tidptr=0x7f2ea15029d0) = 1277
1277  exit(0)                           = ?
1277  +++ exited with 0 +++
1276  exit(0)                           = ?
1276  +++ exited with 0 +++
1274  clone(child_stack=0x7f2ea0d00fb0, flags=CLONE_VM|CLONE_FS|
CLONE_FILES|CLONE_SIGHAND|CLONE_THREAD|CLONE_SYSVSEM|CLONE_SETTLS|
CLONE_PARENT_SETTID|CLONE_CHILD_CLEARTID, parent_tid=[1278], 
tls=0x7f2ea0d01700, child_tidptr=0x7f2ea0d019d0) = 1278
1274  clone(child_stack=0x7f2ea1501fb0, flags=CLONE_VM|CLONE_FS|
CLONE_FILES|CLONE_SIGHAND|CLONE_THREAD|CLONE_SYSVSEM|CLONE_SETTLS|
CLONE_PARENT_SETTID|CLONE_CHILD_CLEARTID <unfinished ...>
1278  exit(0 <unfinished ...>
1274  <... clone resumed>, parent_tid=[1279], tls=0x7f2ea1502700, 
child_tidptr=0x7f2ea15029d0) = 1279
1278  <... exit resumed>)               = ?
1278  +++ exited with 0 +++
1279  exit(0)                           = ?
1279  +++ exited with 0 +++
1274  exit(0)                           = ?
1274  +++ exited with 0 +++
1273  clone(child_stack=0x7f2ea1d02fb0, flags=CLONE_VM|CLONE_FS|
CLONE_FILES|CLONE_SIGHAND|CLONE_THREAD|CLONE_SYSVSEM|CLONE_SETTLS|
CLONE_PARENT_SETTID|CLONE_CHILD_CLEARTID, parent_tid=[1280], 
tls=0x7f2ea1d03700, child_tidptr=0x7f2ea1d039d0) = 1280
1273  clone(child_stack=0x7f2ea0d00fb0, flags=CLONE_VM|CLONE_FS|
CLONE_FILES|CLONE_SIGHAND|CLONE_THREAD|CLONE_SYSVSEM|CLONE_SETTLS|
CLONE_PARENT_SETTID|CLONE_CHILD_CLEARTID <unfinished ...>
1280  clone(child_stack=0x7f2ea1501fb0, flags=CLONE_VM|CLONE_FS|
CLONE_FILES|CLONE_SIGHAND|CLONE_THREAD|CLONE_SYSVSEM|CLONE_SETTLS|
CLONE_PARENT_SETTID|CLONE_CHILD_CLEARTID <unfinished ...>
1273  <... clone resumed>, parent_tid=[1281], tls=0x7f2ea0d01700, 
child_tidptr=0x7f2ea0d019d0) = 1281
1280  <... clone resumed>, parent_tid=[1282], tls=0x7f2ea1502700, 
child_tidptr=0x7f2ea15029d0) = 1282
1281  exit(0)                           = ?
1281  +++ exited with 0 +++
1282  clone(child_stack=0x7f2ea0d00fb0, flags=CLONE_VM|CLONE_FS|
CLONE_FILES|CLONE_SIGHAND|CLONE_THREAD|CLONE_SYSVSEM|CLONE_SETTLS|
CLONE_PARENT_SETTID|CLONE_CHILD_CLEARTID, parent_tid=[1283], 
tls=0x7f2ea0d01700, child_tidptr=0x7f2ea0d019d0) = 1283
1283  exit(0)                           = ?
1283  +++ exited with 0 +++
1282  exit(0)                           = ?
1282  +++ exited with 0 +++
1280  exit(0)                           = ?
1280  +++ exited with 0 +++
1273  write(1, "-454\textbackslash{n}", 5)             = 5
1273  write(1, "-36\textbackslash{n}", 4)              = 4
1273  write(1, "0\textbackslash{n}", 2)                = 2
1273  write(1, "3\textbackslash{n}", 2)                = 2
1273  write(1, "7\textbackslash{n}", 2)                = 2
1273  write(1, "23\textbackslash{n}", 3)               = 3
1273  write(1, "55\textbackslash{n}", 3)               = 3
1273  write(1, "254\textbackslash{n}", 4)              = 4
1273  exit_group(0)                     = ?
1273  +++ exited with 0 +++
\end{alltt}

\pagebreak

\section{Исследование ускорения и эффективности}

Для исследования производительности будем использовать несколько тестов:
на 100000, 1000000, 10000000

Таблица 1. Исследование на 100000 элементах.

\begin{tabular}{|p{35mm}|p{35mm}|p{35mm}|p{35mm}|}
    \hline
    {Количество \newline потоков(n)} &Время работы&{Ускорение \newline $(S_n=T_1/T_n)$}&Эффективность $(X_n=S_n/n)$\\
    \hline
    1&0.137688&-&-\\
    \hline
    2&0.070220&1.96&0.98\\
    \hline
    3&0.074930&1.83&0.61\\
    \hline
    4&0.054423&2.53&0.63\\
    \hline
    5&0.043279&3.18&0.63\\
    \hline
    6&0.051713&2.66&0.44\\
    \hline
    7&0.045668&3.01&0.43\\
    \hline
    8&0.043089&3.19&0.39\\
    \hline
\end{tabular}

Таблица 2. Исследование на 1000000 элементах.

\begin{tabular}{|p{35mm}|p{35mm}|p{35mm}|p{35mm}|}
    \hline
    {Количество \newline потоков(n)} &Время работы&{Ускорение \newline $(S_n=T_1/T_n)$}&Эффективность $(X_n=S_n/n)$\\
    \hline
    1&1.516532&-&-\\
    \hline
    2&0.950804&1.59&0.79\\
    \hline
    3&0.804578&1.88&0.63\\
    \hline
    4&0.562645&2.69&0.67\\
    \hline
    5&0.519941&2.92&0.58\\
    \hline
    6&0.615402&2.46&0.41\\
    \hline
    7&0.623988&2.43&0.35\\
    \hline
    8&0.645662&2.35&0.29\\
    \hline
\end{tabular}

Таблица 3. Исследование на 10000000 элементах.

\begin{tabular}{|p{35mm}|p{35mm}|p{35mm}|p{35mm}|}
    \hline
    {Количество \newline потоков(n)} &Время работы&{Ускорение \newline $(S_n=T_1/T_n)$}&Эффективность $(X_n=S_n/n)$\\
    \hline
    1&32.312667&-&-\\
    \hline
    2&17.034210&1.89&0.95\\
    \hline
    3&17.034277&1.89&0.63\\
    \hline
    4&10.538664&3.07&0.76\\
    \hline
    5&11.227089&2.88&0.57\\
    \hline
    6&11.185609&2.89&0.48\\
    \hline
    7&11.993463&2.69&0.38\\
    \hline
    8&10.383236&3.11&0.39\\
    \hline
\end{tabular}

\pagebreak

\section{Вывод}

Многие языки программирования позволяют пользователю работать с потоками. 
Создание потоков происходит быстрее, чем создание процессов, за счет того, что
при создании потока не копируется область памяти, а они все работают с одной
областью памяти. Поэтому многопоточность используют для ускарения не зависящих
друг от друга, однотипнях задач, которые будут работать параллельно.

Язык Си предоставляет данный функционал пользователям Unix-подобных операционных
систем с помощью библиотеки pthread.h. Средствами языка Си можно совершать системные
запросы на создание и ожидания завершения потока, а также использовать различные
примитивы синхронизации.

В данной лабораторной работе был реализован и исследован алгоритм битонной сортировки.
Установив при этом, что используя 4 потока можно молучить выигрыш по времени
в 2,5-3 раза. При дальнейшем увеличении потоков прирост почти не увеличиватся, а
даже может уменьшаться из-за того, что на управление и переключение потоков 
уходит больше времени, чем они выигрывают.
\end{document}


