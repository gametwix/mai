\section{Выводы}
Выполнив вторую лабораторную работу по курсу <<Дискретный анализ>>, я реализовал Красно-черное дерево.

Сложность вставки и удаления $O(lg(n))$, а максимальная высота корня также $O(lg(n))$,что намного лучше, чем у обычного двоичного дерева.

Главный недостаток в этой структуре это перебаллансировка, что приводит к увеличению сложности алгоритма и времени затраченного на вставку. Тем не менее как раз за счет балансировки мы и получаем бысрое удаление, вставку и поиск.
\pagebreak
