\section{Описание}
Как сказано в \cite{Kormen}: \enquote{Красно-черное дерево представляет собой бинарное дерево поиска с одним дополнительным битом цвета в каждом узле}. Цвет узла либо черный, либо красный. Такое дерево должно удовлетворять следующим свойствам:
\begin{itemize}
	\item Каждый узел является либо красным, либо черным.
	\item Корень дерева является черным узлом.
	\item Каждый лист дерева (Nil) является черным узлом.
	\item Если узел красный, то оба его дочерних узла черные.
\end{itemize}

Вставка и удаление в дереве будет описана далее на примере исходного кода.
\pagebreak

\section{Исходный код}
Данную программу можно разбить на несколько основных частей:
\begin{itemize}
	\item Парсинг введенной команды.
	\item Вставка, удаление и поиск в дереве.
	\item Сохранение в файл и загрузка дерева из файла.
\end{itemize}

Парсинг происходит в два этапа. Сначала мы считываем первое слово, из которого можно определить тип команды. Далее мы смотрим на первый символ этой строки и при помощи switch мы выбираем нужный вариан. Далее, по необходимости, мы считываем дополнительное слово или число. В случае вызова switch default, мы принимаем первое слово с командой как ключ для поиска.

\begin{lstlisting}[language=C++]
class TDict
{
	rb::rb_tree<NPair::TPair> Tree;
	public:
	void parse_comand(NString::TString &cmd)
	{
		NString::TString word;
		NString::TString path;
		NPair::TPair Data;
		bool res;
		switch(cmd[0])
		{
			case '+':
			std::cin >> Data.Str >> Data.Num;
			Data.Str.lower();

			Tree.Search(Data,res);
			if(res == false)
			{
				Tree.insert_data(Data);
				std::cout << "OK" << std::endl;
			}   
			else
				std::cout << "Exist" << std::endl;
			break;
			case '-':
			std::cin >> Data.Str;
			Data.Str.lower();
			Tree.Search(Data,res);
			if(res == true)
			{
				Tree.Delete(Data);
				std::cout << "OK" << std::endl;
			}   
			else
				std::cout << "NoSuchWord" << std::endl;
			
			break;
			case '!':
			std::cin >> word >> path;
			word.lower();
			if(word == "load")
			{
				if(Tree.load(path.str))
					std::cout << "OK" <<std::endl;
				else
					std::cout << "ERROR: can't open file" <<std::endl;
			}
			else
			{
				if(Tree.save(path.str))
					std::cout << "OK" <<std::endl;
				else
					std::cout << "ERROR: can't open file" <<std::endl;
			}
			break;
			default:
			Data.Str = cmd;
			Data.Str.lower();
			rb::rb_tree_elem<NPair::TPair> *elem = Tree.Search(Data,res);
			
			if(res)
			{
				std::cout<< "OK: " << elem->Key.Num << std::endl;
			}
			else
				std::cout << "NoSuchWord" << std::endl;
			break;
		}
	}

};
\end{lstlisting}

Вставка в дерево это немного модифицированная вставка в обычное бинарное дерево поиска. Для того чтобы вставка сохраняла красно-черные свойства дерево используется фенкция ins\_fix(), которая перекрашивает узлы и выполняет повороты.

\begin{lstlisting}[language=C++]
void insert(rb_tree_elem<T> *z)
{
	rb_tree_elem<T> *y = Nil;
	rb_tree_elem<T> *x = Root;
	while(x != Nil)
	{
		y = x;
		if(z->Key < x->Key)
		{
			x = x->Left;
		}
		else
		{
			x = x->Right;
		}
	}
	z->Par = y;
	if(y == Nil)
	{
		Root = z;
	}
	else if(z->Key < y->Key)
	{
		y->Left = z;
	}
	else
	{
		y->Right = z;
	}
	z->Left = Nil;
	z->Right = Nil;
	z->Color = 1;
	
	ins_fix(z);
}
\end{lstlisting}

Исправление вставки можно разделить на 3 случая:
\begin{itemize}
	\item "Дядя" у узла z - красный. Так как z и z.p красный мы красим родителя z и y в черный, а для сохраннения колличества черных узлов мы красим z.p.p в красный, z.p.p становится новым узлом z.
	\item "Дядя" у узла z черный, и z - правый потомок. В случае если мы правый потомок мы поднимаемся на уровень выше и делаем левый поворот. Далее делаем те же действия, что и в случае 3.
	\item "Дядя" у узла z черный, и z - левый потомок.
\end{itemize} 

\newpage
Определение функций:
\begin{longtable}{|p{7.5cm}|p{7.5cm}|}
\hline
\rowcolor{lightgray}
\multicolumn{2}{|c|} {dict.hpp}\\
\hline
void parse\_comand(NString::TString \&cmd&Парсинг запросов\\
\hline
\rowcolor{lightgray}
\multicolumn{2}{|c|} {rb.hpp}\\
\hline
void left\_rotate(rb\_tree \&Tree,rb\_tree\_elem<T> *x)&Левый поворот дерева.\\
\hline
void right\_rotate(rb\_tree \&Tree,rb\_tree\_elem<T> *y)&Правый поворот дерева.\\
\hline
void ins\_fix(rb\_tree\_elem<T> *z)&Исправление вставки в дерево\\
\hline
void insert(rb\_tree\_elem<T> *z)&Вставка в дерево.\\
\hline
void transplant(rb\_tree\_elem<T> *u,rb\_tree\_elem<T> *v)&Переносит элимент в дереве.\\
\hline
void insert\_data(T data)&Создает элимент дерева для вызова вставки.\\
\hline
void print(rb\_tree\_elem<T> *Tree,int lvl)&Вывод дерева.\\
\hline
void clear(rb\_tree\_elem<T> *Tree)&Очистка дерева.\\
\hline
void save\_tree(rb\_tree\_elem<T> *Tree,std::ofstream\& wf)&Сохраняет дерево в файл.\\
\hline
bool save(char *ch)&Проверяет возможность создания и открывает файл на запись.\\
\hline
void load\_tree(rb\_tree\_elem<T> *Tree,std::ifstream\& rf)&Загрузка дерева из файла.\\
\hline
bool load(char *ch)&Проверяет наличие файла и возможность открыть его на чтение.\\
\hline
rb\_tree\_elem<T>* Search(T\& sample,bool\& success)&Поиск элемента в дереве.\\
\hline
rb\_tree\_elem<T>* Tree\_min(rb\_tree\_elem<T>* elem)&Поиск минимального элемента в дереве\\
\hline
void rb\_delete(rb\_tree\_elem<T>* \&elem)&Удаление элемента из дерева.\\
\hline
void delete\_fix(rb\_tree\_elem<T>* x)&Исправление дерева после удаления элемента.\\
\hline
void Delete(T\& sample)&Проверяет дерево на возможность удаления и вызывает удаление.\\
\hline
bool isFileEmpty(const char* filename)&Проверяет закончился ли файл.\\
\hline
\end{longtable}

\newpage
Листинг:
\begin{lstlisting}[language=C]
#pragma once
#include <iostream>
#include <stdio.h>
#include <fstream>

namespace rb
{
	template <typename T>
	class rb_tree_elem
	{
		public:
		bool Color;
		T Key;
		rb_tree_elem *Left;
		rb_tree_elem *Right;
		rb_tree_elem *Par;

		rb_tree_elem()
	};

	template <typename T>
	class rb_tree
	{
		public:
		rb_tree_elem<T> *Root;
		rb_tree_elem<T> *Nil;

		rb_tree()

		~rb_tree()

		void left_rotate(rb_tree &Tree,rb_tree_elem<T> *x)

		void right_rotate(rb_tree &Tree,rb_tree_elem<T> *y)

		void ins_fix(rb_tree_elem<T> *z)

		void insert(rb_tree_elem<T> *z)

		void transplant(rb_tree_elem<T> *u,rb_tree_elem<T> *v)

		void insert_data(T data)

		void print(rb_tree_elem<T> *Tree,int lvl)

		void clear(rb_tree_elem<T> *Tree)

		void save_tree(rb_tree_elem<T> *Tree,std::ofstream& wf)

		bool save(char *ch)

		void load_tree(rb_tree_elem<T> *Tree,std::ifstream& rf)

		bool load(char *ch)

		rb_tree_elem<T>* Search(T& sample,bool& success)

		rb_tree_elem<T>* Tree_min(rb_tree_elem<T>* elem)

		void rb_delete(rb_tree_elem<T>* &elem)

		void delete_fix(rb_tree_elem<T>* x)
		
		void Delete(T& sample)

		bool isFileEmpty(const char* filename)
	};

}//namespace rb
\end{lstlisting}
\pagebreak

\section{Консоль}
\begin{alltt}
	pavel@DESKTOP-VBSMFB3:~/Projects/mai/2_course/DA/LB2/tosol/solution$ cat test0
	+ a 1
	+ A 2
	+ aa 18446744073709551615
	aa
	A
	- A
	a
	pavel@DESKTOP-VBSMFB3:~/Projects/mai/2_course/DA/LB2/tosol/solution$ cat test0 |./solution
	OK
	Exist
	OK
	OK: 18446744073709551615
	OK: 1
	OK
	NoSuchWord
\end{alltt}
\pagebreak

