\section{Описание}
Алогоритм Ахо-Корасик - это алшоритм поиска подстроки в строке. В отсличии от других алгоритмов, алогоритм Ахо-Корасик позволяет деталь поиск сразу несколько паттернов одноврененно, то есть за один проход. Благодаря этому скорость поиска множества паттернов намного выше, чем в других алгоритмах.

Рассмотрим данный алгоритм на примере моего исходного кода.
\pagebreak

\section{Исходный код}
Данный алгоритм можно разбить на несколько основных частей:
\begin{itemize}
	\item Создание trie из паттернов.
	\item Создание суффиксных ссылок.
	\item Поиск в тексте.
\end{itemize}

Добавление паттерна в trie происходит поэлементно. Стартовой позицией является Root, мы смотрим, есть ли у элемента ребенок в виде нашего введенного элемента. Если он существует, то мы просто переходим в него, иначе создаем, добавляем в таблицу детей и переходим. Дойдя до конца паттерна мы говорим, что это последий элемент и добавляем его позицию с учетом джокеров.

\begin{lstlisting}[language=C++]
void Push(const std::vector<long long> &pattern, int pos){
	node_ptr cur = Root;
	for(auto elem_pattern: pattern){
		node_ptr child = cur->ChildPrt(elem_pattern);
		if(child == nullptr){
			child = new TNode(elem_pattern);
			child->Parent = cur;
			cur->Childs.insert({elem_pattern,child});
		}
		cur = child;
	}
	cur->Last = true;
	cur->Jockers.push_back(pos);
}
\end{lstlisting}

Далее происходит создание ссылок на суффиксы и концы префиксов, для этого нужно обойти trie в ширину. При поиске суффикса мы переходим в родителя, и далее перезодим по суффиксам родителя до тех пор пока не у суффикса не будет в качестве ребенка наш елемент, в таком случае это суффикс нащего элемента, иначе идем пока не дойдем до нулевого указателя, в таком случае суффикс нашего элемента это Root. Для проверки на концы префиксов мы смотрим является ли наш суффикс концом паттерна или же указывает ли он другой конец префикса.

\begin{lstlisting}[language=C++]
void BorSuf(){
    //BFS
    std::queue<node_ptr> queue;
    queue.push(Root);
    while(!queue.empty()){
        node_ptr elem = queue.front();
        queue.pop();
        for(auto child: elem->Childs)
        {
            queue.push(child.second);
        }
        //~BFS
        //Suf
        if(elem == Root) //not for Root
		    continue;
        
        node_ptr parent = elem->Parent;
        parent = parent->Suffix;
        while(parent != nullptr && parent->ChildPrt(elem->Sym) == nullptr)
            parent = parent->Suffix;
        
        if(parent == nullptr)
            elem->Suffix = Root;
        else
            elem->Suffix = parent->ChildPrt(elem->Sym);
        //~Suf
        //SubLast
        if(elem->Suffix->Last)
            elem->SubLast = elem->Suffix;
        else if(elem->Suffix->SubLast != nullptr)
            elem->SubLast = elem->Suffix->SubLast;
        //~SubLast
    }//while
}
\end{lstlisting}

При поиске мы смотрим на символ и переходим в ребенка с этим символом, если такого не существует, то переходим по суффиксу пока не сможем перейти или пока не упремся в Root. В случае если мы дости элемента конца паттерна или элемента со ссылок на префикса сонца паттерна, то мы добавляем единицу в вектор вхождений в позиции текста минус позиция паттерна с учетом джокеров. После поиска мы смотрим на вектор вхождений, в точках где значение равно колличеству паттернов и есть начальные точки вохождений паттернов.

\begin{lstlisting}[language=C++]
void ThisLast(node_ptr Cur,std::vector<int> &vect_incl,int pos){
    if(Cur->SubLast != nullptr)
        ThisLast(Cur->SubLast,vect_incl,pos);
    for(auto jocker: Cur->Jockers){ 
        if(pos-jocker >= 0)
            ++vect_incl[pos-jocker];
    }
}

void Find(std::vector<SText> &text,std::vector<int> &pos_incl){
    node_ptr cur = Root;
    int i=0;
    for(auto elem: text){
        while(cur->ChildPrt(elem.Sym)==nullptr && cur->Suffix!=nullptr)
            cur = cur->Suffix;
        if(cur->ChildPrt(elem.Sym)!=nullptr)
            cur = cur->Childs.at(elem.Sym);
        if(cur->Last)
            ThisLast(cur,pos_incl,i);
        else if(cur->SubLast != nullptr)
            ThisLast(cur->SubLast,pos_incl,i);
        
        ++i;
    }
}
\end{lstlisting}

{\bfseries Листинг:}
\begin{lstlisting}[language=C]
#pragma once

#include <vector>
#include <unordered_map>
#include <queue>
#include <iostream>

struct SText
{
	long long Sym;
	int str;
	int word;
};


class TAhoKorasik{
protected:
class TNode{
	public:
	using node_ptr = TNode*;
	long long Sym;                 //Data
	bool Last;
	std::vector<int> Jockers;
	node_ptr Parent;               //Ptrs
	node_ptr Suffix;
	node_ptr SubLast;
	std::unordered_map<long long,node_ptr> Childs;

	TNode(long long inSym = -1,bool inLast = false):Sym(inSym),Last(inLast){
		Parent = nullptr;
		Suffix = nullptr;
		SubLast = nullptr;
	}

	~TNode(){
		for(auto Child: Childs){
			delete(Child.second);
		}
	}

	node_ptr ChildPrt(long long &inSym){
		std::unordered_map<long long,node_ptr>::iterator child = Childs.find(inSym);
		if(child == Childs.end())
			return nullptr;
		else
			return child->second;    
	}
};//TNode
using node_ptr = TNode::node_ptr;

node_ptr Root;

public:

TAhoKorasik(){
	Root = new TNode();
}
~TAhoKorasik(){
	delete(Root);
}

void Push(const std::vector<long long> &pattern, int pos){
	node_ptr cur = Root;
	for(auto elem_pattern: pattern){
		node_ptr child = cur->ChildPrt(elem_pattern);
		if(child == nullptr){
			child = new TNode(elem_pattern);
			child->Parent = cur;
			cur->Childs.insert({elem_pattern,child});
		}
		cur = child;
	}
	cur->Last = true;
	cur->Jockers.push_back(pos);
}

void BorSuf(){
	//BFS
	std::queue<node_ptr> queue;
	queue.push(Root);
	while(!queue.empty()){
		node_ptr elem = queue.front();
		queue.pop();
		for(auto child: elem->Childs)
		{
			queue.push(child.second);
		}
		//~BFS
		//Suf
		if(elem == Root) //not for Root
			continue;
		
		node_ptr parent = elem->Parent;
		parent = parent->Suffix;
		while(parent != nullptr && parent->ChildPrt(elem->Sym) == nullptr)
			parent = parent->Suffix;
		
		if(parent == nullptr)
			elem->Suffix = Root;
		else
			elem->Suffix = parent->ChildPrt(elem->Sym);
		//~Suf
		//SubLast
		if(elem->Suffix->Last)
			elem->SubLast = elem->Suffix;
		else if(elem->Suffix->SubLast != nullptr)
			elem->SubLast = elem->Suffix->SubLast;
		//~SubLast
	}//while
}

void ThisLast(node_ptr Cur,std::vector<int> &vect_incl,int pos){
	if(Cur->SubLast != nullptr)
		ThisLast(Cur->SubLast,vect_incl,pos);
	for(auto jocker: Cur->Jockers){ 
		if(pos-jocker >= 0)
			++vect_incl[pos-jocker];
	}
}

void Find(std::vector<SText> &text,std::vector<int> &pos_incl){
	node_ptr cur = Root;
	int i=0;
	for(auto elem: text){
		while(cur->ChildPrt(elem.Sym)==nullptr && cur->Suffix!=nullptr)
			cur = cur->Suffix;
		if(cur->ChildPrt(elem.Sym)!=nullptr)
			cur = cur->Childs.at(elem.Sym);
		if(cur->Last)
			ThisLast(cur,pos_incl,i);
		else if(cur->SubLast != nullptr)
			ThisLast(cur->SubLast,pos_incl,i);
		
		++i;
	}
}

};//TAhoKorasik
\end{lstlisting}
\pagebreak

\section{Консоль}
\begin{alltt}
pavel@DESKTOP-VBSMFB3:~/Projects/mai/2_course/DA/LB4/second$ cat ../test
1 ? 02
0001
1
02
2
pavel@DESKTOP-VBSMFB3:~/Projects/mai/2_course/DA/LB4/second$ ./solution < ../test
1, 1
2, 1
\end{alltt}
\pagebreak

