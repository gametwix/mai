\documentclass[oneside,a4paper,12pt]{extreport}
\usepackage[utf8]{inputenc}
\usepackage[english,russian]{babel}

\usepackage[a4paper,left=2cm,right=1cm,top=2cm,bottom=2cm]{geometry}


\usepackage{incgraph}
\usepackage{amssymb}
\usepackage{amsmath}
\usepackage{graphicx}
\usepackage{tikz}
\usepackage{wasysym}

\usepackage{fancyhdr}
\pagestyle{fancy}
\renewcommand{\headrulewidth}{0pt} % убираем разделительную линию
\fancyhead{}
\fancyfoot{}
\fancyfoot[C]{\thepage}

\let\oldsection\section

\renewcommand{\section}[2]{\begin{center}\oldsection#1{#2}\end{center}
\addcontentsline{toc}{section}{#2}
}

\newcommand{\includeimage}[3][3mm]{
\vspace{#1}
\includegraphics[scale=#2]{#3}
\vspace{#1}
}

\renewcommand{\theenumi}{\arabic{enumi}.}
\renewcommand{\labelenumi}{\arabic{enumi}.}
\renewcommand{\theenumii}{\arabic{enumii}.}
\renewcommand{\labelenumii}{\arabic{enumi}.\arabic{enumii}.}
\renewcommand{\theenumiii}{\arabic{enumiii}}
\renewcommand{\labelenumiii}{\arabic{enumi}.\arabic{enumii}.\arabic{enumiii}.}


\begin{document}
  {
\thispagestyle{empty}
\begin{center}
\large
Национальный исследовательский университет 

"Московский авиационный институт"

Факультет No8 "Информационные технологии и прикладная математика"

Кафедра 806 "Вычислительная математика и программирование"

\vspace{7cm}
{\bfseries
КУРСОВОЙ ПРОЕКТ

ПО КУРСУ “ДИСКРЕТНАЯ МАТЕМАТИКА”

1 СЕМЕСТР 

“ТЕОРИЯ ГРАФОВ”}

\vspace{9cm}
\end{center}
\begin{flushright}
\begin{tabular}{lp{4cm}lp{4cm}}
Выполнил студент: & Мохляков П. А.\\
Группа: & М80-108Б-19\\
Преподаватель: & Смерчинская С.О.\\
\end{tabular}
\end{flushright}
\vspace{2.2cm}
\begin{center}
Москва 2020
\end{center}
\newpage
}

  \begin{center}\bfseries Итоговый (зачётный) комплекс общеразвивающих упражнений (ОРУ) и растяжки\end{center}

  \begin{flushright}Таблица 1\end{flushright}

  \begin{tabular}{|p{4mm}|p{2.5cm}|p{5.5cm}|p{2cm}|p{2cm}|p{2cm}|}
    \hline
    \textnumero & Название упражнения & Описание упражнения & Дозировка & Рабо\-чие мыш\-цы & Мето\-ди\-чес\-кие ука\-за\-ния\\
    \hline
    1 & Отжимания от поля хватом на уровне плеч & И.П. - упор лежа, руки на ширине плеч. Опустить грудь на уровень 5-10 см от пола. & 3 подхода по 10 повторений & Большие грудные мышцы, трицепсы, дельтовидные  & \\
    \hline
    2 & Отжимания с узким хватом & И.П - упор лежа, руки вместе.Опустить грудь на уровень 5-10 см от пола. & 2 подхода по 10 повторений & Трицепс & \\
    \hline
    3 & "Супермен" & И.П - Лягте на живот, вытянув прямые ноги и руки. Ладони и носки смотрят вниз. Оторвите прямые руки и ноги от пола. Держите корпус неподвижным. Представьте себе Супермена, вытянувшегося в полете. Удерживайте такое положение 15-30 сек, а затем медленно опустите руки и ноги. & 10 повторений & Тра\-пе\-це\-вид\-ные мыш\-цы & \\
    \hline
    4 & Разведение гантелей в наклоне & И.П - Наклонитесь в талии, расставив стопы на ширине плеч, и слегка согнув колени. Отведите таз назад, и втяните живот. Держите гантели, развернув ладони друг к другу, немного согните локти. Руки должны свисать к полу, но не болтаться свободно. Аккуратно поднимайте руки вверх через стороны до параллели с полом. Убедитесь, что локти слегка согнуты, и поднимайте вес силой верхнего отдела спины. & 10 повторений по 30 секундс & Тра\-пе\-це\-вид\-ные мыш\-цы, ши\-ро\-чай\-шие и ром\-бо\-вид\-ные мыш\-цы & \\
    \hline
  \end{tabular}

  \newpage

  \begin{tabular}{|p{4mm}|p{2.5cm}|p{5.5cm}|p{2cm}|p{2cm}|p{2cm}|}
    \hline
    5 & Наклон с поворотом к противоположной стопе & И.П - Станьте прямо, ноги немного шире плеч. Возьмите в каждую руку по снаряду. Если у вас только одна гантель, сожмите ее обеими руками. Вдохните и потянитесь руками вниз к одной стопе, разворачивая к ней корпус. Следите за тем, чтобы колени были немного согнутыми, а руки – прямыми. Наклоняйтесь вперед на комфортную высоту. Медленно выпрямитесь, и нагнитесь к другой стопе. Продолжайте менять стороны. & 20 повторений & Мышцы поясницы & \\
    \hline
    6 & Скручивания с вытянутыми руками & И.П - лежа на полу, руки вытянуты параллельно, продолжают линию туловища, ноги согнуты в коленях. Силой пресса подымай корпус. От пола отрываются голова, руки и ключицы. & 15-20 повторений & Верхний сегмент прямой мышцы & \\
    \hline
    7 & Поднятие ног & И.П - лежа на полу, плотно прижать поясницу. Руки вытягиваются вдоль тела.Приподнять прямую нижнюю конечность, образовав острый угол, удержать, плавно опустить. & 15 повторений & Нижний сегмент прямой мышцы& \\
    \hline
    8 & Косые скручиания & И.П - лягте на спину и согните ноги в коленях. Руки уберите за голову. Тянитесь левым локтем к колену правой ноги, затем правым локтем к колену левой ноги. & По 10 повторений в каждую сторону & Внешние косые мышцы & \\
    \hline
    9 & Поворот прямых ног лежа на спине & И.П - лежа на полу, плотно прижать поясницу. Руки вытягиваются в разные стороны. Ноги подняты. Плавно опускаем ноги сначала в правую, затем в левую сторону не касаясь пола & По 10 повторений в каждую сторону & Внутрение косые мышцы & \\
    \hline
  \end{tabular}
  \newpage
  \begin{tabular}{|p{4mm}|p{2.5cm}|p{5.5cm}|p{2cm}|p{2cm}|p{2cm}|}
    \hline
    10 & Зашагивания на платформу & И.П - Стоим перед платформой ровно, руки опущены, плечи слегка отведены назад. Становимся на платформу вначале одной ногой, затем ставим вторую. & 10-15 повторений & Развитие группы ягодичных мышц, а также передней и задней части поверхности бедра & \\
    \hline
    11 & Положение в присяди у стены & И.П - Становимся спиной к стене и немного от неё отступаем. Медленно опускаемся на воображаемый стул. & 30-60 секунд & Передней стороны бёдер, ягодичные, икроножные & \\
    \hline
    12 & Приседания & И.П - Ноги на ширине плеч, руки вытянуты вперёд, подбородок приподнят. гибая ноги в коленях, опускаемся до положения, при котором в коленном суставе будет образован прямой угол. & 20 повторений, 3 подхода & Внут\-рен\-яя сто\-ро\-на бед\-ра & \\
    \hline
    \multicolumn{6}{|c|}{\bfseries Разтяжка}\\
    \hline
    1 & Растяжка бицепса & И.П - встать прямо, ноги на ширине плеч, сцепить руки за спиной, так чтобы ладони были направлены вниз. В таком положении поднимать руки вверх не сгибая корпс & 2-3 раза & Бицепс & \\
    \hline
    2 & Растяжка трицепса & И.П - положение стоя. Завести руку за голову, свободной рукой взяться за локоть и тянуть к голове & 2-3 раза на руку & Трицепс & \\
    \hline
    3 & Растяжка мышц плеч & И.П - положение стоя. Одну руку завести за спину, вторую за голову и пытаться их соединить в замок & 2-3 раза на руку & Дель\-то\-вид\-ные мыш\-цы & \\
    \hline
    4 & Вытягивание вперед сидя на коленях & И.П -  Сядьте на пол, ягодицы должны касаться пяток. Наклонитесь вперёд, лягте животом на колени и вытяните руки. & 20-30 секунд & Широчайшие и ромбовидные мишцы & \\
    \hline
  \end{tabular}
  \newpage
  \begin{tabular}{|p{4mm}|p{2.5cm}|p{5.5cm}|p{2cm}|p{2cm}|p{2cm}|}
    \hline
    5 & Перевёрнутая растяжка спины & И.П - Лягте на пол на спину, руки вдоль тела, ноги прямые. Поднимите ноги, а затем закиньте их за голову. Руки упираются локтями в пол, кисти поддерживают поясницу. Не опирайтесь на шею, точка опоры — плечи. & 20-30 секунд & Пояснична грудная, широчайшие мышцы & \\
    \hline
    6 & Обратное выгибание \newline (Поза верблюда) & И.П - встаем на колени, беремся рукам за ступни и отталкиваемя вверх. & 2 подхода по 10 секунд & Прямые и косые мышцы & \\
    \hline
    7 & Растяжка лежа на спине & И.П - Ложитесь на спину. Согните одну ногу и тяните её в противоположную для неё сторону, обхватив колено рукой. & Для каждой ноги по 10 секунд. & Косые мышцы & \\
    \hline
    8 & Растягивание мышц прямой ноги & И.П - одна нога прямая, другая согнута в колене и стопа прижата к внутренней поверхности бедра. Притягивая себя к ноге, необходимо тянуться вперёд к стопе, чтобы голова и спина формировали одну линию.  & Для каждой ноги по 10 секунд & Ик\-ро\-нож\-ную, кам\-ба\-ло\-вид\-ную, зад\-нюю боль\-ше\-бер\-цо\-вую мыш\-цу, дву\-гла\-вую мыш\-цу бед\-ра. & \\
    \hline
    9 & Растягивание бедра лёжа на спине & И.П - лягте на спину, согните одну ногу в колене и притяните обеими руками ближе к себе. Необходимо следить, чтобы прямая нога оставалась прижата к полу. & Для каждой ноги по 10 секунд. & Большую ягодичную мышцу & \\
    \hline
    10 & Наклон к прямой ноге с приседанием на другую &  И.П - необходимо вынести одну ногу немного вперёд, другую согнуть в колене, таз увести дальше назад и сохраняя прямую спину тянутся вниз к стопе. Дотянувшись до стопы, начните прижимать себя к ноге, вытягивая весь позвоночник. & Для каждой ноги по 10 секунд. & Мышцы задней поверхности бедра & \\
    \hline
  \end{tabular}
\end{document}
