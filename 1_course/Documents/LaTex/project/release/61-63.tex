
\includeimage{1}{images/61_1.pdf}

И, наконец, мы можем увидеть клетку-победителя, делаем последнее передвижение, и дело сделано.

\vspace{\baselineskip}
\section*{Раздел 12.3: Решаем “Пятнашки” (3х3) используя алгоритм А*}

\pline{0.4}{\textbf{Описание проблемы:}}

“Пятнашки” - это простая игра, представляющая собой сетку размера 3 на 3 (содержащую 9 клеток). Одна из клеток пустая. Задача заключается в передвижении  клеток между собой для получения определенного расположения чисел на этих клетках.

\begin{comment} % тест цветов - оттенков фиолетового
\textcolor{DarkOrchid}{Решаем “Пятнашки” (3х3) используя алгоритм А*}\\
\textcolor{Fuchsia}{Решаем “Пятнашки” (3х3) используя алгоритм А*}\\
\textcolor{Mulberry}{Решаем “Пятнашки” (3х3) используя алгоритм А*}\\
\textcolor{Orchid}{Решаем “Пятнашки” (3х3) используя алгоритм А*}\\
\textcolor{Plum}{Решаем “Пятнашки” (3х3) используя алгоритм А*}\\
\textcolor{RedViolet}{Решаем “Пятнашки” (3х3) используя алгоритм А*}\\
\textcolor{Purple}{Решаем “Пятнашки” (3х3) используя алгоритм А*}\\
\end{comment}

\includeimage{1}{images/61_22.pdf}

Нам же требуется получить из “начального” состояния “итоговое” состояние, используя минимальное количество действий.\\
\pline{0.4}{\textbf{Начальное состояние:}}

\vspace{3mm}

\begin{tcolorbox}
\color{Purple}{
\_ 1 \ 3 
	
4 \ 2 \ 5
	
7 \ 8 \ 6}    
\end{tcolorbox}

\pline{0.4}{\textbf{Итоговое состояние}}

\begin{tcolorbox}
\color{Purple}{
1 \ 2 \ 3 
	
4 \ 5 \ 6
	
7 \ 8 \ \_}
\end{tcolorbox}

\pline{0.4}{\textbf{Предполагаемая эвристика}}

Предположим, что расстояние городских кварталов(Manhattan distance) между текущим и финальным состоянием является эвристикой, требуемой для данной задачи.

\begin{tcolorbox}
    h(n) = | x - p | + | y - q |
\end{tcolorbox}

где x и y координаты в текущем состоянии

\hspace{6mm} p и q координаты в финальном состоянии

\pline{0.4}{\textbf{Функция конечной стоимости}}

Функция конечной стоимости , полученная,

\begin{tcolorbox}
    f(n) = g(n) + h(n), где g(n) стоимость необходимая для достижения текущего состояния из заданного начального состояния
\end{tcolorbox}

\pline{0.4}{\textbf{Решение задачи}}

Сначала найдем значение эвристики, требуемое для получения итогового состояния из начального. Функция стоимости g(n) = 0, пока мы находимся в начальном состоянии.

\begin{tcolorbox}
    {h(n) = 8}
\end{tcolorbox}

Значение выше показывает, что “1” в текущем состоянии смещена на одну клетку по горизонтали от своего положения в финальном состоянии. Аналогично происходит для “2”, ”5”, ”6”. \_ отличается от своего итогового положения на 2 клетки по горизонтали и на 2 по вертикали. Таким образом, общее значение для f(n) получается: 1 + 1 + 1 + 1 + 2 + 2 = 8. Функция конечной стоимости равна 8 + 0 = 8.

~

Теперь, когда все возможные состояния, которые могут быть получены из исходного, найдены, мы видим, что можем переместить \_ вправо или вниз.

~

Состояния, которые можно получить, выполнив доступные действия выглядят следующим образом:

\begin{tcolorbox}
\color{Purple}{
1 \ \hspace{-0.5mm}\_ \hspace{0.5mm}3 \hspace{1.cm} 4 \ 1 \ 3
	
4 \ 2 \ 5 \hspace{1.05cm}\_ \ \hspace{-0.5mm}2 \ 5
	
7 \ 8 \ 6  \hspace{0.99cm} 7 \ 8 \ 6

\hspace{3.5mm}(1)\hspace{19mm}(2)}
\end{tcolorbox}

Снова функция стоимости для этих состояний рассчитывается с использованием метода, описанного выше, и оказывается в итоге равной 6 и 7 соответственно. Мы выбираем состояние с наименьшей стоимостью, и им оказывается состояние 1. Следующими возможными шагами могут быть: шаг влево, шаг вправо или шаг вниз. Мы не станем делать шаг влево, так как мы пришли из этого состояния. Поэтому мы движемся либо вниз, либо вправо.

~

Опять находим состояния, которые могут быть получены из состояния (1).

\begin{tcolorbox}
\color{Purple}{
1 \ 3 \ \hspace{-0.5mm}\_ \hspace{0.9cm} 1 \ 2 \ 3
	
4 \ 2 \ 5 \hspace{0.94cm} 4 \ \hspace{-0.2mm}\_ \hspace{-0.3mm}\ 5
	
7 \ 8 \ 6  \hspace{0.98cm} 7 \ 8 \ 6

\hspace{3.4mm}(3)\hspace{18.9mm}(4)}
\end{tcolorbox}

(3) приводит к получению значения 6 , а (4) приводит к значению 4 в функции стоимости. Также будем считать , что состояние (2), найденное ранее, имеет стоимость 7. Выбираем состояние с минимальной стоимостью из всех возможных. И им становится состояние (4). Следующими возможными шагами могу быть шаги влево, вправо или вниз.

\begin{tcolorbox}
\color{Purple}{
1 \ 2 \ 3 \hspace{0.98cm} 1 \ 2 \ 3 \hspace{0.98cm} 1 \ 2 \ 3
	
\hspace{-0.5mm}\_\hspace{-0.6mm} \ 4 \ 5 \hspace{0.98cm} 4 \ 5 \ \hspace{-0.6mm}\_\hspace{1.05cm} 4 \ 8 \ 5
	
7 \ 8 \ 6  \hspace{0.98cm} 7 \ 8 \ 6 \hspace{0.98cm}  4 \ \hspace{-0.4mm}\_ \hspace{-0.7mm}\ 5

\hspace{3.4mm}(5)\hspace{18.9mm}(6)\hspace{19.1mm}(7)}

\end{tcolorbox}

Получаем стоимости равные 5, 2 и 4 для (5), (6) и (7) соответственно. Также у нас есть предыдущие состояния (3) и (2) со стоимость 6 и 7 соответственно. Выбираем состояние с наименьшей стоимостью - состояние (6). Следующие возможные шаги: вверх и вниз. И именно шаг вниз приведет нас к финальному состоянию, приравнивая значение эвристической функции к 0.



