% Моисеенков

Если алгоритм не находит соответствия, он просто возвращает \textbf{-1}.

\vspace{\baselineskip}

Этот алгоритм используется при обнаружении плагиата. Исходя из исходного материала, алгоритм может быстро искать в документе примеры предложений из исходного материала, игнорируя такие детали, как регистр и знаки препинания. Из-за обилия искомых строк алгоритмы поиска в одной строке здесь нецелесообразны. Опять же, \textbf{алгоритм Кнута-Морриса-Пратта} или \textbf{алгоритм поиска строк Бойера-Мура} являются более быстрыми алгоритмами поиска по одной строке шаблона, чем \textbf{алгоритм Рабина-Карпа}.Тем не менее, это алгоритм выбора для поиска по нескольким шаблонам. Если мы хотим найти любое из большого числа, скажем, k, шаблонов фиксированной длины в тексте, мы можем создать простой вариант алгоритма Рабина-Карпа.

\vspace{\baselineskip}

Для текста шаблонов длины \textbf{n} и \textbf{p} общей длины \textbf{m} его среднее и наилучшее время выполнения равно \textbf{O(n+m)} в пространстве \textbf{O(p)}, но его наихудшее время равно \textbf{O(nm)}.

\section*{Раздел 40.3: Реализация алгоритма КМП на Python}

\textbf{Стог сена}: Строка, в которой необходимо найти данный шаблон. \\
\textbf{Игла}: Шаблон для поиска.

\vspace{\baselineskip}

\textbf{Временная сложность}: Часть поиска (метод strstr) имеет сложность O(n), где n - длина стога сена, но для построения таблицы префиксов также требуется анализ иглы, поэтому сложность O(m) требуется для построения таблицы префиксов, где m - длина иглы. \\
Следовательно, общая временная сложность для КМП составляет \textbf{O(n+m)}. \\
\textbf{Пространственная сложность}: \textbf{O(m)} из-за таблицы префиксов на игле.

\vspace{\baselineskip}

Примечание: Следующая реализация возвращает начальную позицию совпадения в стоге сена (если есть совпадение), в противном случае возвращается -1, для крайних случаев, например, если игла или стог сена является пустой строкой или игла не найдена в стоге сена.

\begin{tcolorbox}
\begin{minted}{Python}
def get_prefix_table(needle):   
	prefix_set = set()    
	n = len(needle)   
	prefix_table = [0]*n  
	delimeter = 1
	while(delimeter<n):        
		prefix_set.add(needle[:delimeter])        
		j = 1        
		while(j<delimeter+1):           
			if needle[j:delimeter+1] in prefix_set:
				prefix_table[delimeter] = delimeter - j + 1       
				break            
			j += 1        
		delimeter += 1 
	return prefix_table

def strstr(haystack, needle):    
	haystack_len = len(haystack)    
	needle_len = len(needle)    
	if (needle_len > haystack_len) or (not haystack_len) or (not needle_len):        
		return -1    
	prefix_table = get_prefix_table(needle)    
	m = i = 0    
	while((i<needle_len) and (m<haystack_len)):        
		if haystack[m] == needle[i]:            
			i += 1
			m += 1        
		else:            
			if i != 0:               
				i = prefix_table[i-1]            
			else:                
				m += 1    
	if i==needle_len and haystack[m-1] == needle[i-1]:        
		return m - needle_len    
	else:        
		return -1

if __name__ == '__main__':    
	needle = 'abcaby'    
	haystack = 'abxabcabcaby'    
	print strstr(haystack, needle)
\end{minted}
\end{tcolorbox}

\section*{Раздел 40.4: Алгоритм КМП на Си}

С учетом текста \textit{txt} и шаблона \textit{pat}, целью этой программы будет распечатать все вхождения \textit{pat} в \textit{txt}.

\vspace{\baselineskip}

\textbf{Примеры}:

\vspace{\baselineskip}

\textbf{Входные данные}:

\vspace{\baselineskip}

\begin{tcolorbox}
\begin{minted}{C}
txt[] = "THIS IS A TEST TEXT"
pat[] = "TEST"
\end{minted}
\end{tcolorbox}

\vspace{\baselineskip}

\textbf{Выходные данные}:

\vspace{\baselineskip}

\begin{tcolorbox}
\begin{minted}{C}
Pattern found at index 10
\end{minted}
\end{tcolorbox}

\vspace{\baselineskip}

\textbf{Входные данные}:

\vspace{\baselineskip}

\begin{tcolorbox}
\begin{minted}{C}
txt[] = "AABAACAADAABAAABAA"
pat[] = "AABA"
\end{minted}
\end{tcolorbox}

\vspace{\baselineskip}

\textbf{Выходные данные}:

\vspace{\baselineskip}

\begin{tcolorbox}
\begin{minted}{C}
Pattern found at index 0
Pattern found at index 9
Pattern found at index 13
\end{minted}
\end{tcolorbox}

\vspace{\baselineskip}

\textbf{Реализация на языке Си}:

\begin{tcolorbox}
\begin{minted}{C}
// Программа на языке Си для реализации алгоритма КМП поиска 
// подстроки 
#include<stdio.h> 
#include<string.h> 
#include<stdlib.h>
void computeLPSArray(char *pat, int M, int *lps);

void KMPSearch(char *pat, char *txt) 
{   
	int M = strlen(pat);    
	int N = strlen(txt);

 
	int *lps = (int *)malloc(sizeof(int)*M);    
	int j  = 0; 

	    
	computeLPSArray(pat, M, lps);

	int i = 0;     
	while (i < N)    
	{      
		if (pat[j] == txt[i])      
		{        
			j++;        
			i++;      
		}

		if (j == M)      
		{        
			printf("Found pattern at index %d \n", i-j);        
			j = lps[j-1];      
		}

   
		else if (i < N && pat[j] != txt[i])      
		{        
			if (j != 0)         
				j = lps[j-1];        
			else         
				i = i+1;      
		}    
	}    
	free(lps);
}

void computeLPSArray(char *pat, int M, int *lps) 
{    
	int len = 0;    
	int i;

	lps[0] = 0   
	i = 1;

	while (i < M)    
	{       
		if (pat[i] == pat[len])       
		{         
			len++;         
			lps[i] = len;         
			i++;       
		}       
		else     
		{         
			if (len != 0)         
			{          
				         
				len = lps[len-1];

			}         
			else       
			{          
				lps[i] = 0;           
				i++;         
			}       
		}    
	} 
}

int main() 
{   
	char *txt = "ABABDABACDABABCABAB";   
	char *pat = "ABABCABAB";   
	KMPSearch(pat, txt);   
	return 0; 
}
\end{minted}
\end{tcolorbox}

\textbf{Выходные данные}:

\vspace{\baselineskip}

\begin{tcolorbox}
\begin{minted}{c}
Found pattern at index 10
\end{minted}
\end{tcolorbox}

\vspace{\baselineskip}

\textbf{Ссылка}:

\vspace{\baselineskip}

\href{https://www.geeksforgeeks.org/kmp-algorithm-for-pattern-searching/}{\underline{https://www.geeksforgeeks.org/kmp-algorithm-for-pattern-searching/}}

\chapter*{Глава 41: Поиск в ширину}
\section*{Раздел 41.1. Нахождение кратчайшего пути от источника к другим узлам}

\href{https://vk.cc/9pBLeT}{\underline{Поиск в ширину}}(BFS) - это алгоритм обхода или поиска таких структур данных, как дерево или граф. Он начинается с корня дерева (или некоторого произвольного узла графа, иногда называемого «ключом поиска») и в первую очередь исследует соседние узлы в первую очередь, прежде чем перейти к соседям следующего уровня. BFS был изобретен в конце 1950-х годов \href{https://en.wikipedia.org/wiki/Edward_F._Moore}{\underline{Эдвардом Форрестом Муром}}, который использовал его, чтобы найти кратчайший путь из лабиринта и, независимо с К. Ю. Ли, обнаружил, что его можно использовать в качестве алгоритма проводной маршрутизации в 1961 году.

\vspace{\baselineskip}

Процесс работы алгоритма BFS описан в этих предложениях:

\begin{enumerate}
    \item Мы не пройдем ни один узел более одного раза.
    \item Исходный узел или узел, с которого мы начинаем, расположен на уровне 0.
    \item Узлы, к которым мы можем напрямую обратиться из узла-источника, являются узлами уровня 1, узлы, к которым мы можем напрямую обратиться из узлов уровня 1, являются узлами уровня 2 и так далее.
    \item Уровень обозначает расстояние по кратчайшему пути от источника.
\end{enumerate}

Давайте посмотрим на пример:

\begin{center}
    \includeimage{0.7}{images/195_1.pdf}
\end{center}

Давайте предположим, что этот граф представляет связь между несколькими городами, где каждый узел означает город, а ребро между двумя узлами означает, что существует дорога, связывающая их. Мы хотим перейти от \textbf{узла 1} к \textbf{узлу 10}. Таким образом, \textbf{узел 1} является нашим \textbf{источником}, который является \textbf{уровнем 0}. Мы помечаем \textbf{узел 1} как посещенный. Отсюда мы можем перейти к \textbf{узлу 2}, \textbf{3} и \textbf{4}. Таким образом, они будут \textbf{уровнями (0 + 1)} = \textbf{уровнями 1}. Теперь мы отметим их как посещенные и будем работать с ними.

\begin{center}
    \includeimage{0.7}{images/196_1.pdf}
\end{center}

Окрашенные узлы посещаются. Узлы, с которыми мы сейчас работаем, отмечаются розовым цветом. Мы не будем посещать один и тот же узел дважды. Из \textbf{узлов 2}, \textbf{3} и \textbf{4}, мы можем пойти к \textbf{узлам 6}, \textbf{7} и \textbf{8}. Давайте отметим их как посещенные. Уровень этих узлов будет \textbf{уровень (1 + 1)} = \textbf{уровень 2}.

\begin{center}
    \includeimage{0.7}{images/197_1.pdf}
\end{center}

Если вы не заметили, уровень узлов просто обозначает кратчайшее расстояние от \textbf{источника}. Например: мы нашли \textbf{узел 8} на \textbf{уровне 2}. Таким образом, расстояние от \textbf{источника} до \textbf{узла 8} равно \textbf{2}.

\vspace{\baselineskip}

Мы еще не достигли нашего целевого узла, то есть \textbf{узла 10}. Итак, давайте посетим следующие узлы. Мы можем напрямую перейти от \textbf{узла 6}, \textbf{7} и \textbf{узла 8}.

\begin{center}
    \includeimage{0.7}{images/198_1.pdf}
\end{center}

Мы видим, что мы нашли \textbf{узел 10} на \textbf{уровне 3}. Таким образом, кратчайший путь из \textbf{источника} к \textbf{узлу 10} равен \textbf{3}. Мы исследовали граф уровень за уровнем и нашли кратчайший путь. Теперь давайте удалим рёбра, которые мы не использовали:

\begin{center}
    \includeimage{0.7}{images/199_1.pdf}
\end{center}

После удаления ребер, которые мы не использовали, мы получаем дерево, называемое BFS деревом. Это дерево показывает кратчайший путь от \textbf{источника} ко всем остальным узлам.

\vspace{\baselineskip}

Таким образом, наша задача состоит в том, чтобы перейти от узла \textbf{источника} к узлам \textbf{уровня 1}. Затем с \textbf{уровня 1} до \textbf{уровня 2} и так далее, пока мы не достигнем пункта назначения. Мы можем использовать \textit{очередь} для хранения узлов, которые мы собираемся обработать. То есть для каждого узла, с которым мы собираемся работать, мы будем добавлять в очередь все другие узлы, которые могут быть напрямую пройдены или которые еще не пройдены в очереди.

\vspace{\baselineskip}

Моделирование нашего примера:

\vspace{\baselineskip}

Сначала мы помещаем источник в очередь. Наша очередь будет выглядеть так:

\vspace{\baselineskip}

\begin{tcolorbox}
\hspace{3mm}front \\
{\tiny{+ - - - - - +}}

\hspace{0.4mm}|\hspace{4.2mm}1\hspace{4.2mm}|

{\tiny{+ - - - - - +}}
\end{tcolorbox}

\vspace{\baselineskip}

Уровень \textbf{узла 1} будет равен 0. \textbf{level[1] = 0}. Теперь мы начинаем наш BFS. Сначала мы извлекаем узел из нашей очереди. Мы получаем \textbf{узел 1}. Мы можем перейти к \textbf{узлу 4}, \textbf{3} и \textbf{2} от него. Мы достигли этих узлов из \textbf{узла 1}. Таким образом, l\textbf{evel[4] = level[3] = level[2] = level[1] + 1 = 1}. Теперь мы помечаем их как посещенных и помещаем их в очередь.

\vspace{\baselineskip}

\begin{tcolorbox}
\hspace{3.35cm}front \\
{\tiny{+ - - - - - +} \ \tiny{+ - - - - - +} \ \tiny{+ - - - - - +}}

\hspace{0.4mm}|\hspace{4.2mm}2\hspace{4.2mm}| \ |\hspace{4.2mm}3\hspace{4.2mm}| \ |\hspace{4mm}4\hspace{4mm}|

{\tiny{+ - - - - - +} \ \tiny{+ - - - - - +} \ \tiny{+ - - - - - +}}
\end{tcolorbox}

\vspace{\baselineskip}

Теперь мы извлекаем \textbf{узел 4} и работаем с ним. Мы можем перейти к \textbf{узлу 7} из \textbf{узла 4}. \textbf{level[7] = level[4] + 1 = 2}. Мы помечаем \textbf{узел 7} как посещенный и помещаем его в очередь.

\vspace{\baselineskip}

\begin{tcolorbox}
\hspace{3.35cm}front \\
{\tiny{+ - - - - - +} \ \tiny{+ - - - - - +} \ \tiny{+ - - - - - +}}

\hspace{0.4mm}|\hspace{4.2mm}7\hspace{4.2mm}| \ |\hspace{4.2mm}2\hspace{4.2mm}| \ |\hspace{4mm}3\hspace{4mm}|

{\tiny{+ - - - - - +} \ \tiny{+ - - - - - +} \ \tiny{+ - - - - - +}}
\end{tcolorbox}

\vspace{\baselineskip}

Из \textbf{узла 3} мы можем перейти к \textbf{узлу 7} и \textbf{узлу 8}. Так как мы уже пометили \textbf{узел 7} как посещенный, мы пометим теперь \textbf{узел 8}. Мы меняем \textbf{level[8] = level[3] + 1 = 2} и добавляем \textbf{узел 8} в очередь.

\begin{tcolorbox}
\hspace{3.35cm}front \\
{\tiny{+ - - - - - +} \ \tiny{+ - - - - - +} \ \tiny{+ - - - - - +}}

\hspace{0.4mm}|\hspace{4.2mm}6\hspace{4.2mm}| \ |\hspace{4.2mm}7\hspace{4.2mm}| \ |\hspace{4mm}2\hspace{4mm}|

{\tiny{+ - - - - - +} \ \tiny{+ - - - - - +} \ \tiny{+ - - - - - +}}
\end{tcolorbox}

\vspace{\baselineskip}

Этот процесс будет продолжаться до тех пор, пока мы не достигнем пункта назначения или очередь не станет пустой. Массив \textbf{уровней} предоставит нам расстояние по кратчайшему пути от \textbf{источника}. Мы можем инициализировать массив \textbf{уровней} значением \textit{бесконечность}, которое отметит, что узлы еще не посещены. Наш псевдокод будет выглядеть так:

\begin{tcolorbox}
\begin{minted}{C}
Procedure BFS(Graph, source): 
Q = queue(); 
level[] = infinity 
level[source] := 0 
Q.push(source) 
while Q is not empty    
	u -> Q.pop()    
	for all edges from u to v in Adjacency list        
		if level[v] == infinity            
			level[v] := level[u] + 1            
			Q.push(v)        
		end if    
	end for 
end while 
Return level
\end{minted}
\end{tcolorbox}

Перебирая массив \textbf{уровней}, мы можем узнать расстояние от источника до каждого узла. Например: расстояние до \textbf{узла 10} от \textbf{источника} будет сохранено в \textbf{level[10]}.

\vspace{\baselineskip}

Иногда нам может потребоваться распечатать не только кратчайшее расстояние, но и путь, по которому мы можем перейти к нашему конечному узлу из \textbf{источника}. Для этого нам нужно сохранить массив \textbf{parent}. \textbf{parent[source]} будет равен NULL. Для каждого обновления в массиве \textbf{level} мы просто добавим {\usefont{T2A}{cmtt}{m}{n} parent[v] := u} в наш псевдокод внутри цикла for. После завершения BFS, чтобы найти путь, мы будем перемещаться назад по массиву \textbf{parent}, пока не достигнем \textbf{источника}, который будет обозначен значением NULL. \\
Псевдокод будет выглядеть так:

\begin{tcolorbox}
\begin{minted}{C}
//рекурсивный                               //итеративный
Procedure PrintPath(u):                 |   Procedure PrintPath(u):   
if parent[u] is not equal to null       |   S =  Stack()
    PrintPath(parent[u])                |   while parent[u] is not equal to null                                   equal to null
end if                                  |       S.push(u) 
print -> u                              |       u := parent[u]
                                        |   end while
                                        |   while S is not empty
                                        |       print -> S.pop
                                        |   end while

\end{minted}
\end{tcolorbox}

\textbf{Сложность}:

\vspace{\baselineskip}

Мы посетили каждый узел и каждое ребро один раз. Таким образом, сложность будет \textbf{O(V+E)}, где \textbf{V} - количество узлов, а \textbf{E} - количество ребер.

\section*{Раздел 41.2. Поиск кратчайшего пути от источника в двумерном графе}

Большую часть времени нам нужно будет искать кратчайший путь от одного источника ко всем остальным узлам или к конкретному узлу в 2D-графе. Скажем, например, что мы хотим выяснить, сколько ходов требуется, чтобы конь достиг определенного квадрата на шахматной доске, или у нас есть массив, в котором некоторые ячейки заблокированы, и мы должны найти кратчайший путь от одной ячейки к другой. Мы можем двигаться только горизонтально и вертикально. Однако диагональные ходы могут быть тоже возможны. Для этих случаев мы можем преобразовать квадраты или ячейки в узлы и легко решить эту проблему, используя BFS. Теперь наши \textbf{visited}, \textbf{parents} и \textbf{level} будут двумерными массивами. Для каждого узла мы рассмотрим все возможные ходы. Чтобы найти расстояние до определенного узла, мы также проверим, достигли ли мы пункта назначения.

\vspace{\baselineskip}

Будет еще один массив, называемый массивом направлений. Он просто сохранит все возможные комбинации направлений, по которым мы можем идти. Допустим, для горизонтальных и вертикальных перемещений наши массивы направлений будут:

\begin{tcolorbox}
{\tiny{+ - - - - - + - - - - - + - - - - - + - - - - - + - - - - - +}}

\hspace{0.4mm}|\hspace{3mm}dx\hspace{3mm}|\hspace{4.2mm}1\hspace{4.2mm}|\hspace{3.6mm}-1\hspace{3.6mm}|\hspace{4.2mm}0\hspace{4.2mm}|\hspace{4.2mm}0\hspace{4.2mm}|

{\tiny{+ - - - - - + - - - - - + - - - - - + - - - - - + - - - - - +}}

\hspace{0.4mm}|\hspace{3mm}dy\hspace{3mm}|\hspace{4.2mm}0\hspace{4.2mm}|\hspace{4.2mm}0\hspace{4.2mm}|\hspace{4.2mm}1\hspace{4.2mm}|\hspace{3.6mm}-1\hspace{3.6mm}|

{\tiny{+ - - - - - + - - - - - + - - - - - + - - - - - + - - - - - +}}
\end{tcolorbox}

Здесь \textit{dx} представляет собой движение по оси x, а \textit{dy} - движение по оси y. Опять же эта часть не обязательна. Вы также можете написать все возможные комбинации отдельно, но легче справиться с этим, используя массив направлений. Комбинации для диагональных ходов и ходов коня могут быть разнообразней, а их количество гораздо больше.

\vspace{\baselineskip}

Дополнительная часть, на которую мы должны обратить внимание:

\begin{itemize}
    \item Если какая-либо из ячеек заблокирована, то для каждого возможного хода мы будем проверять, заблокирована ли ячейка или нет.
    \item Мы также проверим, вышли ли мы за пределы, то есть пересекли ли мы границы массива.
    \item Количество строк и столбцов будет дано.
\end{itemize}

Наш псевдокод будет выглядеть так:

\begin{tcolorbox}
\begin{minted}{C}
Procedure BFS2D(Graph, blocksign, row, column): 
for i from 1 to row    
	for j from 1 to column        
		visited[i][j] := false    
	end for 
end for 
visited[source.x][source.y] := true 
level[source.x][source.y] := 0 
Q = queue() 
Q.push(source) 
m := dx.size 
while Q is not empty    
	top := Q.pop    
	for i from 1 to m
		temp.x := top.x + dx[i]        
		temp.y := top.y + dy[i]        
		if temp is inside the row and column and top 
		                        does not equal to blocksign
			visited[temp.x][temp.y] := true            
			level[temp.x][temp.y] := level[top.x][top.y] + 1            
			Q.push(temp)
        end if
    end for
end while
Return level
\end{minted}
\end{tcolorbox}