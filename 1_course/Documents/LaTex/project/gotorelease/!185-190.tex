%Арапов%
\chapter*{Глава 40: Поиск подстроки}
\section*{Раздел 40.1: Введение в алгоритм Кнута-Морриса-Пратта (KMП)}
Предположим, что у нас есть \textit{текст} и \textit{шаблон}. Нам нужно определить, существует ли шаблон в тексте
или нет. Например:
\vspace{\baselineskip}
\begin{tcolorbox}\begin{verbatim}
+-------+---+---+---+---+---+---+---+---+
| Index | 0 | 1 | 2 | 3 | 4 | 5 | 6 | 7 |
+-------+---+---+---+---+---+---+---+---+
|  Text | a | b | c | b | c | g | l | x |
+-------+---+---+---+---+---+---+---+---+
+---------+---+---+---+---+
| Index   | 0 | 1 | 2 | 3 |
+---------+---+---+---+---+
| Pattern | b | c | g | l |
+---------+---+---+---+---+
\end{verbatim} 
\end{tcolorbox}
\vspace{\baselineskip}
Этот шаблон \textit{существует} в \textit{тексте}. Таким образом, наш поиск по подстроке должен возвращать \textbf{3}, индекс
позиции, с которой начинается этот \textit{шаблон}. Так как же работает наша процедура поиска подстрок
грубой силы?
\vspace{\baselineskip}

Что мы обычно делаем: начинаем с \textbf{0-го} индекса текста и \textbf{0-го} \textit{индекса} нашего *шаблона и сравниваем
\textbf{Text[0]} (текст) с \textbf{Pattern[0]} (шаблон). Поскольку они не совпадают, мы переходим к следующему
индексу нашего \textit{текста} и сравниваем \textbf{Text[1]} с \textbf{Pattern[0]}. Поскольку это совпадение, мы увеличиваем
индекс нашего \textit{шаблона} и индекс \textit{текста}. Мы сравниваем \textbf{Text[2]} с \textbf{Pattern[1]}. Они также совпадают.
Следуя процедуре, описанной ранее, теперь мы сравним \textbf{Text[3]} с \textbf{Pattern[2]}. Поскольку они не
совпадают, мы начинаем со следующей позиции, где начали искать совпадение. Это \textbit{индекс} \textbf{2} текста. Мы
сравниваем \textbf{Text[2]} с \textbf{Pattern[0]}. Они не совпадают. Затем, увеличивая индекс \textit{текста}, мы сравниваем
\textbf{Text[3]} с \textbf{Pattern[0]}. Они совпадают. Снова \textbf{Text[4]} и \textbf{Pattern[1]} совпадают, \textbf{Text[5]} и \textbf{Pattern[2]}
совпадают, \textbf{Text[6]}и \textbf{Pattern[3]} совпадают. Поскольку мы достигли конца нашего \textit{шаблона}, мы
возвращаем индекс, с которого началось наше совпадение, то есть \textbf{3}. Если наш \textit{шаблон} был: bcgll %Непонятный шрифт%
, это
означает, что если \textit{шаблон} не существует в нашем \textit{тексте}, наш поиск должен вернуть исключение или \textbf{-1}
или любое другое предопределенное значение. Мы можем ясно видеть, что в худшем случае этот
алгоритм будет занимать O (mn) %непонятный шрифт%
времени, где \textbf{m} - длина \textit{текста}, а \textbf{n} - длина \textit{шаблона}. Как уменьшить это
время? Вот тут-то и появляется алгоритм поиска подстроки KMП.
\vspace{\baselineskip}

\href{https://en.wikipedia.org/wiki/Knuth%E2%80%93Morris%E2%80%93Pratt_algorithm
}
{\underline {Алгоритм поиска строк Кнута-Морриса-Пратта}}
или алгоритм KMП ищет вхождения «шаблона» в
основном «тексте», используя наблюдение о том, что при возникновении несоответствия само слово
содержит достаточно информации, чтобы определить, где может начаться следующее совпадение ,
таким образом, минуя пересмотр ранее согласованных символов.\newline Алгоритм был задуман в 1970 году
\href{https://en.wikipedia.org/wiki/Donald_Knuth}{\underline{Дональдом Кнутом}}
и \href{https://en.wikipedia.org/wiki/Vaughan_Pratt}{\underline{Воном Праттом}} и независимо \newline\href{https://en.wikipedia.org/wiki/James_H._Morris}{\underline{Джеймсом Х. Моррисом}}.
Трио опубликовало его совместно в 1977 году.
\vspace{\baselineskip}

Давайте расширим наш пример \textit{Text}(текста) и \textit{Pattern}(шаблона) для лучшего понимания:
\vspace{\baselineskip}
\begin{tcolorbox}
\begin{verbatim}
+-------+--+--+--+--+--+--+--+--+--+--+--+--+--+--+--+--+--+--+--+--+--+--+--+
| Index |0 |1 |2 |3 |4 |5 |6 |7 |8 |9 |10|11|12|13|14|15|16|17|18|19|20|21|22|
+-------+--+--+--+--+--+--+--+--+--+--+--+--+--+--+--+--+--+--+--+--+--+--+--+
|  Text |a |b |c |x |a |b |c |d |a |b |x |a |b |c |d |a |b |c |d |a |b |c |y |
+-------+--+--+--+--+--+--+--+--+--+--+--+--+--+--+--+--+--+--+--+--+--+--+--+
+---------+---+---+---+---+---+---+---+---+
|  Index  | 0 | 1 | 2 | 3 | 4 | 5 | 6 | 7 |
+---------+---+---+---+---+---+---+---+---+
| Pattern | a | b | c | d | a | b | c | y |
+---------+---+---+---+---+---+---+---+---+
\end{verbatim}
\end{tcolorbox}
\vspace{\baselineskip}
Сначала наши \textit{Text} и \textit{Pattern} совпадают до индекса \textbf{2}. \textbf{Text[3]} и \textbf{Pattern[3]} не совпадают. Поэтому наша
цель - не возвращаться назад в этом \textit{тексте}, то есть в случае несоответствия мы не хотим, чтобы наше

2
сопоставление начиналось снова с той позиции, с которой мы начали сравнение. Чтобы достичь этого,
мы будем искать \textbf{суффикс} в нашем \textit{шаблоне} прямо перед тем, как произойдет наше несовпадение
(подстрока \textbf{abc}), которая также является \textbf{префиксом} подстроки нашего \textit{шаблона}. В нашем примере,
поскольку все символы уникальны, суффикса нет, это префикс нашей совпавшей подстроки. Так что это
означает, что наше следующее сравнение начнется с индекса \textbf{0}. Подожди немного, скоро ты поймешь,
почему мы это сделали. Далее мы сравниваем \textbf{Text[3]} с \textbf{Pattern[0]}, и он не совпадает. После этого для
\textit{Text} с индекса \textbf{4} по индекс \textbf{9} и для \textit{Pattern} с индекса \textbf{0} по индекс \textbf{5} мы находим совпадение. Мы находим
несоответствие в \textbf{Text[10]} и \textbf{Pattern[6]}. Таким образом, мы берем подстроку из \textit{Pattern} прямо перед
местом, где происходит несовпадение (подстрока \textbf{abcdabc}), мы проверяем суффикс, который также
является префиксом этой подстроки. Здесь мы видим, что \textbf{ab} является как суффиксом, так и префиксом
этой подстроки. Это означает, что, поскольку мы сопоставили до \textbf{Text[10]}, символами, стоящими прямо
перед несоответствием, являются \textbf{ab}. Из этого мы можем сделать вывод, что, поскольку \textbf{ab} также
является префиксом подстроки, которую мы взяли, нам не нужно снова проверять \textbf{ab}, и следующая
проверка может начинаться с \textbf{Text[10]} и \textbf{Pattern[2]}. Нам не нужно было оглядываться назад на весь \textit{Text},
мы можем начать непосредственно с того места, где произошло наше несоответствие. Теперь мы
проверяем \textbf{Text[10]} и \textbf{Pattern[2]}, так как это несоответствие, а подстрока перед несоответствием (\textbf{abc}) не
содержит суффикса, который также является префиксом, мы проверяем \textbf{Text[10]} и \textbf{Pattern[0]}, они не
совпадают. После этого для \textit{Text} начиная с индекса \textbf{11} и до индекса \textbf{17} и для \textit{Pattern} с индекса \textbf{0} и до
индекс \textbf{6}. Мы находим несоответствие в \textbf{Text[18]} и \textbf{Pattern[7]}. Итак, еще раз мы проверяем подстроку
перед несовпадением (substring \textbf{abcdabc}) и находим \textbf{abc} как суффикс, так и префикс. Таким образом,
поскольку у нас совпадало до \textbf{Pattern[7]}, \textbf{abc} должен быть перед \textbf{Text[18]}. Это означает, что нам не
нужно сравнивать до \textbf{Text[17]}, и наше сравнение начнется с \textbf{Text[18]} и \textbf{Pattern[3]}. Таким образом, мы
найдем совпадение и вернем \textbf{15}, которое является нашим начальным индексом совпадения. Вот как
работает наш поиск по подстроке KMП с использованием информации о суффиксах и префиксах.
\vspace{\baselineskip}

Теперь поговорим о том, как нам эффективно вычислить, если суффикс совпадает с префиксом, и в
какой момент начать проверку, если есть несоответствие символа между \textit{текстом} и \textit{шаблоном}. Давайте
посмотрим на пример:
\vspace{\baselineskip}
\begin{tcolorbox}
\begin{verbatim}
+---------+---+---+---+---+---+---+---+---+
|  Index  | 0 | 1 | 2 | 3 | 4 | 5 | 6 | 7 |
+---------+---+---+---+---+---+---+---+---+
| Pattern | a | b | c | d | a | b | c | a |
+---------+---+---+---+---+---+---+---+---+
\end{verbatim}
\end{tcolorbox}
\vspace{\baselineskip}

Мы сгенерируем массив, содержащий необходимую информацию. Назовем этот массив \textbf{S}. Размер
массива будет таким же, как длина шаблона. Поскольку первая буква \textit{шаблона (Pattern)} не может быть
суффиксом какого-либо префикса, мы положим \textbf{S[0] = 0}. Сначала мы берем \textbf{i = 1} и \textbf{j = 0}. На каждом шаге
мы сравниваем \textbf{Pattern[i]} и \textbf{Pattern[j]} и увеличиваем \textbf{i}. Если есть совпадение, мы помещаем в \textbf{S[i] = j + 1}
и увеличиваем \textbf{j}, если есть несоответствие, то мы проверяем предыдущую позицию значения \textbf{j} (если
доступно) и устанавливаем \textbf{j = S[j-1]} (если \textbf{j} не равно \textbf{0}), мы продолжаем делать это до тех пор, пока \textbf{S[j]}
не совпадет с \textbf{S[i]} или \textbf{j} не станет равно \textbf{0}. Для более позднего мы положим \textbf{S[i] = 0}. Для нашего примера:
\vspace{\baselineskip}
\begin{tcolorbox}
\begin{verbatim}
            j   i
+---------+---+---+---+---+---+---+---+---+
|  Index  | 0 | 1 | 2 | 3 | 4 | 5 | 6 | 7 |
+---------+---+---+---+---+---+---+---+---+
| Pattern | a | b | c | d | a | b | c | a |
+---------+---+---+---+---+---+---+---+---+
\end{verbatim}
\end{tcolorbox}
\vspace{\baselineskip}

\textbf{Pattern[j]} и \textbf{Pattern[i]} не совпадают, поэтому мы увеличиваем \textbf{i} и, поскольку \textbf{j} равно \textbf{0}, мы не проверяем
предыдущее значение и присваиваем \textbf{Pattern [i] = 0}. Если мы продолжаем увеличивать \textbf{i}, для \textbf{i = 4} мы
получим совпадение, поэтому мы присваиваем \textbf{S[i] = S[4] = j + 1 = 0 + 1 = 1} и увеличиваем \textbf{j} и \textbf{i}. Наш массив будет выглядеть так:
\vspace{\baselineskip}
\begin{tcolorbox}
\begin{verbatim}
                j               i
+---------+---+---+---+---+---+---+---+---+
|  Index  | 0 | 1 | 2 | 3 | 4 | 5 | 6 | 7 |
+---------+---+---+---+---+---+---+---+---+
| Pattern | a | b | c | d | a | b | c | a |
+---------+---+---+---+---+---+---+---+---+
|    S    | 0 | 0 | 0 | 0 | 1 |   |   |   |
+---------+---+---+---+---+---+---+---+---+
\end{verbatim}

\end{tcolorbox}
\vspace{\baselineskip}
Поскольку \textbf{Pattern[1]} и \textbf{Pattern[5]} совпадают, мы присваиваем \textbf{S[i] = S[5] = j + 1 = 1 + 1 = 2}. Если мы
продолжим, мы найдем несоответствие для \textbf{j = 3} и \textbf{i = 7}. Поскольку \textbf{j} не равно \textbf{0}, присваиваем j = S[j-1].
И мы сравним символы в \textbf{i} и \textbf{j} одинаковые они или нет, так как они одинаковые, мы присваиваем \textbf{S[i] = j
+ 1}. Наш законченный массив будет выглядеть так:
\vspace{\baselineskip}
\begin{tcolorbox}
\begin{verbatim}
+---------+---+---+---+---+---+---+---+---+
|    S    | 0 | 0 | 0 | 0 | 1 | 2 | 3 | 1 |
+---------+---+---+---+---+---+---+---+---+
\end{verbatim}
\end{tcolorbox}
\vspace{\baselineskip}
Это наш обязательный массив. В нем ненулевое значение \textbf{S[i]} означает, что существует суффикс длины
\textbf{S[i]}, такой же, как и префикс в этой подстроке (подстрока от \textbf{0} до \textbf{i}), и следующее сравнение начнется с
\textbf{S[i] + 1} позиции \textit{Pattern}. Наш алгоритм генерации массива будет выглядеть так:
\vspace{\baselineskip}
\begin{tcolorbox}
\begin{minted}{C} 
Procedure GenerateSuffixArray(Pattern):
i :=1
j :=0
n := Pattern.length
while i is less than n
    if Pattern[i] is equal to Pattern[j]
        S[i]:= j +1
        j := j +1
        i := i +1
        else
            if j is not equal to 0
                j := S[j-1]else
                S[i]:=0 
                i := i +1
        end if 
    end if
end while
\end{minted}
%пропущен код%
\end{tcolorbox}
\vspace{\baselineskip}
Сложность по времени для построения этого массива составляет O(n)%непонятный шрифт%
, а сложность пространства также
равна O(n)%непонятный шрифт%
. Чтобы убедиться, что вы полностью поняли алгоритм, попробуйте сгенерировать массив для
шаблона \textbf{aabaabaa} и проверить, совпадает ли результат с \href{https://i.stack.imgur.com/4aqZk.jpg}{\underline{этим}}.
\vspace{\baselineskip}

Теперь давайте выполним поиск по подстроке, используя следующий пример:
\vspace{\baselineskip}
\begin{tcolorbox}
\begin{verbatim}
+---------+---+---+---+---+---+---+---+---+---+---+---+---+
|  Index  | 0 | 1 | 2 | 3 | 4 | 5 | 6 | 7 | 8 | 9 |10 |11 |
+---------+---+---+---+---+---+---+---+---+---+---+---+---+
|   Text  | a | b | x | a | b | c | a | b | c | a | b | y |
+---------+---+---+---+---+---+---+---+---+---+---+---+---+
+---------+---+---+---+---+---+---+
|  Index  | 0 | 1 | 2 | 3 | 4 | 5 |
+---------+---+---+---+---+---+---+
| Pattern | a | b | c | a | b | y |
+---------+---+---+---+---+---+---+
|    S    | 0 | 0 | 0 | 1 | 2 | 0 |
+---------+---+---+---+---+---+---+
\end{verbatim}
\end{tcolorbox}
\vspace{\baselineskip}
У нас есть \textit{Text}, \textit{Pattern} и предварительно рассчитанный массив \textbf{S} с использованием нашей
логики, представленной ранее. Мы сравниваем \textbf{Text[0]} и \textbf{Pattern[0]}, и они одинаковы.
\textbf{Text[1]} и \textbf{Pattern[1]} одинаковы. \textbf{Text[2]} и \textbf{Pattern[2]} не совпадают.
Мы проверяем значение в позиции
прямо перед несоответствием. Поскольку \textbf{S[1]} равно \textbf{0}, нет суффикса, совпадающего с
префиксом в нашей подстроке, и наше сравнение начинается с позиции \textbf{S[1]}, которая равна \textbf{0}.Таким образом, \textbf{Pattern[0]} не совпадает с \textbf{Text[2]}, поэтому мы идем дальше. Значение \textbf{Text[3]}
такой же, как и значение \textbf{Pattern[0]}, а значит есть совпадение до \textbf{Text[8]} и \textbf{Pattern[5]}
соответственно. Мы делаем на шаг назад в массив \textbf{S} и находим \textbf{2}.
Таким образом, это означает,
что существует префикс длины \textbf{2}, который также является суффиксом этой подстроки (\textbf{abcab}),
который является ab. Это также означает, что перед Text[8] стоит ab. Таким образом, мы
можем спокойно игнорировать Pattern[0] и Pattern[1] и начать наше следующее сравнение с
Pattern[2] и Text[8]. Если продолжать дальше, мы найдем образец в тексте. Наша процедура
будет выглядеть так:
\vspace{\baselineskip}
\begin{tcolorbox}
\begin{minted}{C}
Procedure KMP(Text, Pattern)
GenerateSuffixArray(Pattern)
m := Text.Length
n := Pattern.Length
i :=0
j :=0
while i is less than m
    if Pattern[j] is equal to Text[i]
        j := j +1
        i := i +1
        if j is equal to n
            Return(j-i)
        elseif i < m and Pattern[j] is not equal t Text[i]
            if j is not equal to 0
                j = S[j-1]
            else
                i := i +1
        end if
    end if
end while
Return-1
\end{minted}
\end{tcolorbox}
\vspace{\baselineskip}
Временная сложность этого алгоритма, за исключением вычисления суффиксного массива,
составляет O(m). Поскольку \textit{GenerateSuffixArray} принимает O(n), общая временная сложность
алгоритма KMП составляет: O(m + n).
\vspace{\baselineskip}

PS: Если вы хотите найти несколько вхождений Pattern (шаблона) в Text (текст), вместо того,
чтобы возвращать значение, выведете его / сохраните и установите j: = S [j-1]. Также
сохраняйте флажок, чтобы отслеживать, нашли ли вы какое-либо вхождение или нет, и
обрабатывайте его соответствующим образом.

\section*{Раздел 40.2: Введение в алгоритм Рабина-Карпа}
\href{https://en.wikipedia.org/wiki/Rabin%E2%80%93Karp_algorithm
}{\underline{Алгоритм Рабина-Карпа}}
- это алгоритм поиска строк, созданный \href{https://en.wikipedia.org/wiki/Richard_M._Karp}{\underline{Ричардом М. Карпом}} и \newline
\href{https://en.wikipedia.org/wiki/Michael_O._Rabin}{\underline{Майклом О. Рабином}}, который использует хеширование для поиска любого из набора
шаблонных строк в тексте.
\vspace{\baselineskip}

Подстрока - это другая строка, которая встречается в строке. Например, \textit{ver} является
подстрокой \textit{stackoverflow}. Не путать с подпоследовательностью, потому что \textit{cover} - это
подпоследовательность той же строки. Другими словами, любое подмножество
последовательных букв в строке является подстрокой данной строки.
\vspace{\baselineskip}

В алгоритме Рабина-Карпа мы сгенерируем хеш нашего шаблона, который мы ищем, и
проверим, совпадает ли хэш нашего текста с шаблоном или нет. Если он не совпадает, мы
можем гарантировать, что шаблон не существует в тексте.Однако, если он совпадает, \textit{pattern} \textbf{может} присутствовать в тексте. Давайте посмотрим на пример:
\vspace{\baselineskip}

Допустим, у нас есть текст: \textbf{yeminsajid}, и мы хотим выяснить, существует ли в тексте шаблон
\textbf{nsa}. Чтобы вычислить хеш и скользящий хеш, нам нужно использовать простое число. Это
может быть любое простое число. Давайте возьмем \textbf{простое число = 11} для этого примера. Мы
определим значение хеша, используя эту формулу:
\vspace{\baselineskip}
\begin{tcolorbox}
$(1st~letter)~X~(prime)+(2nd~etter)~ X~ (prime)^1~+~(3rd~letter)~ X~ (prime)^2~ X~ + ......$
\end{tcolorbox}
\vspace{\baselineskip}
Мы будем обозначать:
\vspace{\baselineskip}
\begin{tcolorbox}
\begin{tabbing}
MMMMMMMM \= MMMMMMMM \= MMMMMMMM \= MMMMMMMM \= MMMMMMMM \kill
a -> \p{1}\>g -> \p{7}\>m -> \p{13}\>s -> \p{19}\>y -> \p{25}\\
b -> \p{2}\>h -> \p{8}\>n -> \p{14}\>t -> \p{20}\>z -> \p{26}\\
c -> \p{3}\>i -> \p{9}\>o -> \p{15}\>u -> \p{21}\\
d -> \p{4}\>j -> \p{10}\>p -> \p{16}\>v -> \p{22}\\
e -> \p{5}\>k -> \p{11}\>q -> \p{17}\>w -> \p{23}\\
f -> \p{6}\>l -> \p{12}\>r -> \p{18}\>x -> \p{24}
\end{tabbing}
\end{tcolorbox}
\vspace{\baselineskip}
Значение хеша \textbf{nsa} будет:
\vspace{\baselineskip}
\begin{tcolorbox}
$ \p{14}~ X~  \p{11}^0~+~ \p{19}~ X~  \p{11}^1~+~ \p{1}~ X~  \p{11}^2~=~ \p{344}$
\end{tcolorbox}
\vspace{\baselineskip}
Теперь мы находим скользящий хэш нашего текста. Если скользящий хеш совпадает со
значением хэша нашего шаблона, мы проверим соответствуют строки или нет. Поскольку наш
шаблон состоит из \textbf{3} букв, мы возьмем первые \textbf{3} буквы \textbf{yem} из нашего текста и вычислим
значение хеша. Мы получили:
\vspace{\baselineskip}
\begin{tcolorbox}
$\p{25}~ X~ \p{11}^0~+~\p{5}~ X~ \p{11}^1~+~\p{13}~ X~ \p{11}^2~=~\p{1653}$
\end{tcolorbox}
\vspace{\baselineskip}
Это значение не совпадает со значением хэша нашего шаблона. Так что строка здесь не
существует. Теперь нам нужно рассмотреть следующий шаг для вычисления значения хеша
нашей следующей строки \textbf{emi}. Мы можем рассчитать это по нашей формуле. Но это было бы
довольно тривиально и стоило бы нам дороже. Вместо этого мы используем другую технику.

\vspace{\baselineskip}
\begin{itemize}
    \item Мы вычитаем значение \textbf{первой буквы предыдущей строки} из нашего текущего значения хеша. В этом случае \textbf{y}. Мы получаем, $\p{1653} - \p{25} = \p{1628}$.
    \item Мы делим разницу с нашим \textbf{простым числом}, которое в данном примере равно \textbf{11}. Получаем, $\p{1628}/\p{11} = \p{148}$.
    \item Мы добавляем \textbf{новую букву X (простое число) −1}, где \textbf{m} - длина шаблона, с частным, которое равно \textbf{i = 9}. Получаем, $\p{148} + \p{9} X \p{11}^2 = \p{1237}$.
\end{itemize}
\vspace{\baselineskip}

Новое хеш-значение не равно нашему хеш-значению шаблона. Двигаясь дальше,для \textbf{n} мы
получаем:
\vspace{\baselineskip}
\begin{tcolorbox}
\begin{minted}{C}
Previous String: emi
First Letter of Previous String: e(5)
New Letter: n(14)
NewString:"min"
1237 - 5 = 1232
1232 / 11 = 112
112 + 14 X 11^2 = 1806
\end{minted}
\end{tcolorbox}
\vspace{\baselineskip}
Это не соответствует. После этого для \textbf{s} получаем:
\vspace{\baselineskip}
\begin{tcolorbox}
\begin{minted}{C}
Previous String: min
First Letter of Previous String: m(13)
New Letter: s(19)
NewString:"ins"
1806 - 13 = 1793
1793 / 11 = 163
163 + 19 X 11^2 = 2462
\end{minted}
\end{tcolorbox}
\vspace{\baselineskip}
Это не соответствует. Далее, для \textbf{a}, мы получаем:
\vspace{\baselineskip}
\begin{tcolorbox}
\begin{minted}{C}
Previous String: ins
First Letter of Previous String: i(9)
New Letter: a(1)
NewString:"nsa"
2462 - 9 = 2453
2453 / 11 = 223
223 + 1 X 11^2 = 344
\end{minted}
\end{tcolorbox}
\vspace{\baselineskip}
Совпало! Теперь мы сравним наш шаблон с текущей строкой. Поскольку обе строки
совпадают, то подстрока существует в этой строке. И мы возвращаем стартовую позицию
нашей подстроки.
\vspace{\baselineskip}

Псевдокод:
\vspace{\baselineskip}

\textit{Расчет хеша:}
\vspace{\baselineskip}
\begin{tcolorbox}
\begin{minted}{C}
Procedure Calculate-Hash(String, Prime, x):
hash :=0        
for m from 1 to x                           hash value
    hash := hash +(Value of String[m])^(-1)
end for
Return hash
\end{minted}
\end{tcolorbox}
\vspace{\baselineskip}

\textit{Пересчет хеша:}

\vspace{\baselineskip}
\begin{tcolorbox}
\begin{minted}{C}
Procedure Recalculate-Hash(String, Curr, Prime, Hash):
Hash := Hash - Value of String[Curr]     First Letter of Previous String
Hash := Hash / Prime
m :=String.length
New:= Curr + m -1
Hash := Hash +(Value of String[New])^(-1)
Return Hash
\end{minted}
\end{tcolorbox}
\vspace{\baselineskip}

\textit{Строка соответствия:}

\vspace{\baselineskip}
\begin{tcolorbox}
\begin{minted}{C}
Procedure String-Match(Text, Pattern, m):
for i from m to Pattern-length + m -1
    if Text[i] is not equal to Pattern[i]
        Return false
    end if
end for
Return true
\end{minted}
\end{tcolorbox}
\vspace{\baselineskip}

\textit{Rabin-Karp:}

\vspace{\baselineskip}
\begin{tcolorbox}
\begin{minted}{C}
Procedure Rabin-Karp(Text, Pattern, Prime):
m := Pattern.Length
HashValue := Calculate-Hash(Pattern, Prime, m)
CurrValue := Calculate-Hash(Pattern, Prime, m)
for i from 1 to Text.length- m
    if HashValue == CurrValue and String-Match(Text, Pattern, i) is true
        Return i
    end if    
    CurrValue := Recalculate-Hash(String, i+1, Prime, CurrValue)
end for
\end{minted}
\end{tcolorbox}
\vspace{\baselineskip}