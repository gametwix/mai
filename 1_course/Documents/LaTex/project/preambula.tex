% !TeX spellcheck = russian-aot
\documentclass[12pt,openany]{book}
\usepackage[T2A]{fontenc}
\usepackage{sectsty}
%\usepackage{berasans}
%\renewcommand{\sfdefault}{berasans}
%\renewcommand{\familydefault}{\sfdefault}
\usepackage{blindtext}



\usepackage[russian]{babel}
\usepackage[a4paper,left=1.5cm,right=1.5cm,top=1cm,bottom=2cm,bindingoffset=0cm,headsep=5mm]{geometry}
\usepackage[dvipsnames]{xcolor}
\usepackage[most]{tcolorbox} % для управления цветом фона

\usepackage{minted} % подсветка синтаксиса
\usemintedstyle{emacs}
%\usemintedtabsize{5}

\usepackage{hyperref}
\definecolor{urlcolor}{HTML}{7a328b} % цвет гиперссылок b262c6
\definecolor{linkcolor}{HTML}{7a328b} % цвет ссылок
\hypersetup{pdfstartview={ FitH}, linkcolor=linkcolor, urlcolor=urlcolor, colorlinks=true}

\usepackage{fancyhdr}
\pagestyle{fancy}
\renewcommand{\headrulewidth}{0pt} % убираем разделительную линию
\renewcommand{\footrulewidth}{1pt}
\fancyhead{}
\fancyfoot{}
\futurelet\TMPfootrule\def\footrule{{\color{Purple}\TMPfootrule}} % закрашиваем линию
\fancyfoot[R]{\textcolor{Purple}{\thepage}} % закрашиваем номера страниц
\fancyfoot[L]{\href{https://goalkicker.com/}{GoalKicker.com - Алгоритмы. Заметки для Профессионалов}} % создаем гиперссылку

\fancypagestyle{plain} % настройка контитула страниц с chapter-ом 

\usepackage{colortbl}
\usepackage{eurosym}
\usepackage{incgraph}
\usepackage{amssymb}
\usepackage{amsmath}
\usepackage{graphicx}
\usepackage{tikz}

\usepackage{rotating}
\usepackage{systeme}

\usepackage{comment}


\usepackage{sectsty}
\chapterfont{\vspace{-2cm}\color{Purple}}  % sets colour of chapters
\sectionfont{\vspace{-0.5cm}\color{Purple}}  % sets colour of sections


\graphicspath{ {./images/} } 

\setlength{\parindent}{0em}

\newcommand{\pline}[3][0mm] % розовая линия с жирным текстом сбоку
{
\vspace{#1}
~~~~
\begin{minipage}{0.02\linewidth}
    \begin{tikzpicture}
        \draw[line width=0.5mm,Purple](0,0)--(0,#2); 
    \end{tikzpicture}
\end{minipage}
\begin{minipage}{0.94\linewidth}
    {#3} 
\end{minipage}\hfill
\vspace{0mm}
}

\newcommand{\includeimage}[3][3mm]{
\vspace{#1}
\includegraphics[scale=#2]{#3}
\vspace{#1}
}

\newcommand{\p}[1]{\textcolor{Purple}{#1}}

\tikzset
{
    x=2cm,% default value is 1cm.
    y=3cm,% default value is 1cm.
}



\tcbset{colback=gray!5!white, enhanced jigsaw, breakable, boxrule=0mm, arc=3mm, top=2mm, left=1.5mm, bottom=1mm} % параметры серого прямоугольника


\usepackage{minted}
\usemintedstyle{emacs}

%\pagestyle{empty} % нумерация выкл.
\setcounter{page}{1}
\pagestyle{fancy} % нумерация вкл.

\let\oldchapter\chapter
\let\oldsection\section
%
\usepackage{titlesec}
\titlespacing*{\oldchapter}{0pt}{-60pt}{20pt}


\usepackage{soul}
\renewcommand{\chapter}[2]{
\oldchapter#1{#2}\vspace{-0.5cm}
\addcontentsline{toc}{chapter}{\texorpdfstring{\underline{#2}}{#2}}
}

\renewcommand{\section}[2]{\oldsection#1{#2}
\addcontentsline{toc}{section}{\texorpdfstring{\underline{#2}}{#2}}
}
\usepackage{tocloft}
\renewcommand\cftchapdotsep{\cftdot}
\renewcommand{\cftchapleader}{\cftdotfill{\cftchapdotsep}}
\renewcommand\cftsecdotsep{\cftdot}
\renewcommand{\cftchapfont}{\bfseries}
\renewcommand{\cfttoctitlefont}{\Huge\bfseries\color{Purple}}
\renewcommand{\cftchappagefont}{\normalfont}
%

\setminted[python]{obeytabs=true,tabsize=3}
\setminted[Python]{obeytabs=true,tabsize=3}
\setminted[java]{obeytabs=true,tabsize=3}
\setminted[Java]{obeytabs=true,tabsize=3}
\setminted[C]{obeytabs=true,tabsize=3}
\setminted[c]{obeytabs=true,tabsize=3}
\setminted[C++]{obeytabs=true,tabsize=3}
\setminted[c++]{obeytabs=true,tabsize=3}
\setminted[csharp]{obeytabs=true,tabsize=3}
\setminted[Csharp]{obeytabs=true,tabsize=3}
\setminted[CSharp]{obeytabs=true,tabsize=3}
\setminted[OCaml]{obeytabs=true,tabsize=3}
\setminted[ocaml]{obeytabs=true,tabsize=3}
\setminted[Pascal]{obeytabs=true,tabsize=3}
\setminted[pascal]{obeytabs=true,tabsize=3}

