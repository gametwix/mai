\section*{Задание 5}

Найти остовное дерево с минимальной суммой длин входящих в него ребер.

\vspace{\baselineskip}

\begin{center}
  \includeimage{1.2}{images/5_1.pdf}
\end{center}

%\vspace{\baselineskip}

{\large\textbf{Решение:}}

\begin{enumerate}
  \item Выбираем все вершины графа
  \item Добавляем все дуги, имеюще минимальный вес --- 1. Циклов нет.
  \item Добавляем все дуги, имеюще минимальный вес --- 2. Циклов нет.
  \item Добавляем все дуги, имеюще минимальный вес --- 3. Циклов нет.
  \item Добавляем все дуги, имеюще минимальный вес --- 4. Циклов нет.
  \item Добавляем все дуги, имеюще минимальный вес --- 5, так, чтобы не было циклов. Получаем три возможных вариантов деревьев.
  \item Добавляем все дуги, имеюще минимальный вес --- 6. Нет вариантов без циклов.
  \item Добавляем все дуги, имеюще минимальный вес --- 7. Циклов нет.
  \item Добавляем все дуги, имеющие минимальный вес --- 8, так, чтобы не было циклов. Получаем четыре возможных вариантов остовных деревьев минимального веса. Минимальный вес остовного дерева L(D) = 46.
\end{enumerate}

Возможные остовные деревья с минимальной суммой длин ребер - 46:

\vspace{\baselineskip}

\begin{tabular}{ccc}
  \includeimage{0.8}{images/5_2.pdf} & \includeimage{0.8}{images/5_3.pdf} & \includeimage{0.8}{images/5_4.pdf}
\end{tabular}
