\section*{Задание 8(Индивидуальное)}

Построение графа группы по образующим и определяющим соотношениям.

\begin{center}
  \large
  \vspace{\baselineskip}
  \bfseries Теоритическая часть

  %\vspace{\baselineskip}
  \textit{Введение}
\end{center}

Задача о построении графа группы по образующим -- это задача построения графа по базовой ниформации о них. Данная задача является NP-трудной задачей, то есть не существует алгоритма для ее решения за полиномимальное время. Это утвержение верно, если $P \neq NP$. Это связанно с тем, что эта задача влкючает в себя задачу определения равенства двух элимнтов, которая также является NP-трудной. Поэтому данную задачу невозможно решить за полимилиальное время, а только полным передором, для определения равенства элиментов группы.

Если бы данная задача была полимилиальной, то построение графа до пятого колена (ограничение для на циклических групп) имело бы сожность $ \frac{n^5-1}{n-1}$.

\vspace{\baselineskip}


\begin{center}
  \textbf{\textit{Алгоритм решения}}
\end{center}

\begin{enumerate}
  \item Создаем все возможные правила замены на основе определяющих соотношений.

  \item Сортирем правила замены подстрок от замены набольшей подстроки на менименьшую, до замены наименьшей на наибольшую.

  \item Рекурсивно добавляем образующие к уже существующим элиментам.

  \item Проверяем равен ли полученный элимент одному из пред идущих.

    \begin{enumerate}

      \item Переносим один из элиментов на левую строну ($ab^{-1}$).

      \item Заменяем подстроки из списка замен, пока элимент не станет равен нулю или пока не останется строка не имеющая построк из списка замен.

      \item Если полученный элимент пуст, то два введенных элимента равны, иначе они различны.

    \end{enumerate}

  \item Получаем массив вершин графа.

  \item Добавляем по образующей к каждому элименту, и сравнивем его со всеми элиментами массива, если полученный элимент равен одному из элиментов массива, то между данными вершинами графа есть ребро.

  \item На основе полученных данных получаем матрицу смежности графа.

\end{enumerate}

\newpage

\begin{center}
  \textbf{\textit{Логическая блок-схема алгоритма}}
\end{center}

\includeimage{1}{images/8_1.pdf}

\newpage

Вычислительная часть программы написана на C, так как это один из наиболее быстрых языков, что важно, так как задача решается за не полимилиальное время. Интерфейс написан на Python с использованиеми библиотек PyQt и Networkx. В таком случае программа быстро работает и позваляет легко расширять возможности интерфейса.

\begin{center}
  \textbf{\textit{Оценка сложности алгоритма}}
\end{center}

Самый быстрый способ завершить программу в том случае, если все вершины не равны, так как процесс сравнения элиментов группы занимает наиббольшее время, но в таком случае группа точно не является циклической, поэтому мы не идем дальше пятого потомка. В этом лучшем случае программа имеет сложность $ \frac{n^5-1}{n-1}$. Во всех иных случаях программы приходит к решению за полимилиальное время.

\begin{center}
  \textbf{\textit{Пример работы. Скриншоты программы}}
\end{center}

Рассмотрим работу программы на обном из примеров.

Для примера возьмем циклическую группу с двумя образующими:$r$,$f$, и тремя определяющими соотношениями: $r^3=e$,$f^2=e$,$fr=r^2f$.

На основе этих определяющих соотношений мы получаем правила замены:

\begin{tabular}{l}
  $fr^{-2}=e$ \\
  $fr^{-1}=r$ \\
  $f=r^2$ \\
  $r^2f^{-1}=e$ \\
  $r^2=f$ \\
  $r=fr^{-1}$\\
\end{tabular}

Сортируем полученные правила:

\begin{tabular}{l}
  $fr^{-2}=e$ \\
  $fr^{-1}=r$ \\
  $r^2f^{-1}=e$ \\
  $f=r^2$ \\
  $r^2=f$ \\
  $r=fr^{-1}$\\
\end{tabular}

Пооцередно добаляем образующие к элименту и сравниваем его с остальными вершинами графа, пробуя сократить полученный элимент на основе полученных правил замены. Таким образом мы получаем список вершин графа:

\begin{tabular}{l}
  $r$ \\
  $e$ \\
  $r^2$ \\
  $r^3$ \\
  $r^4$ \\
  $r^5$ \\
  $r^5f$ \\
  $r^4f$ \\
  $r^4f^2$ \\
\end{tabular}

\newpage

Добавляем по образующей к каждому элименту, и сравнивем его со всеми элиментами массива, если полученный элимент равен одному из элиментов массива, то между данными вершинами графа есть ребро. Таким образом мы можем построить матрицу смежности:

\vspace{\baselineskip}

\begin{tabular}{*{9}{c}}
  0 & 0 & 1 & 1 & 0 & 0 & 0 & 0 & 0 \\
  1 & 0 & 1 & 0 & 0 & 0 & 0 & 0 & 0 \\
  0 & 0 & 0 & 1 & 1 & 0 & 0 & 0 & 0 \\
  0 & 0 & 0 & 0 & 1 & 1 & 0 & 0 & 0 \\
  0 & 0 & 0 & 0 & 0 & 1 & 0 & 1 & 0 \\
  0 & 0 & 0 & 0 & 0 & 0 & 1 & 1 & 0 \\
  0 & 0 & 0 & 0 & 0 & 0 & 0 & 0 & 1 \\
  0 & 0 & 0 & 0 & 0 & 0 & 1 & 0 & 1 \\
  0 & 0 & 0 & 0 & 0 & 0 & 0 & 0 & 0 \\
\end{tabular}

На основе полученной матрицы смежности мы можем построить граф. Для этого рассмотрим скриншоты.

\includeimage{0.25}{images/8_2.png}

\begin{center}
  \textbf{\textit{Пример прикладной задачи}}
\end{center}

Проверка на наличие подгрупп, и нахождение всех элиментов группы.

\newpage

\begin{center}
  \large{\textbf{Список использованных ресурсов}}
\end{center}

\begin{enumerate}
  \item И.Гросман В.Магнус. Группы и их графы. Мир, 1971.

  \item Берж К. Теория графов и её применение. М.: Изд-во Иностранной литературы, 1962.-320с.

  \item Гроссман И., Магнус В. Группы и их графы. Пер. с англ. Г.М. ЦукерманПод ред. В.Е. Тараканова М. Мир, 1971. -231 с.

  \item Емеличев В.А., Мельников О.И., Сарванов В.И., Тышкевич Р.И. Лекции по теории графов. М.: Либроком, 2009.

  \item Кофман А. Введение в прикладную комбинаторику. М.: Наука, 1975.

  \item Н. Кристофидес: Теория графов. Алгоритмический подход. -М.: МИР, 1978.
\end{enumerate}
