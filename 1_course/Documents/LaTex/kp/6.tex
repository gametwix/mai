\section*{Задание 6}

Пусть каждому ребру неориентированного графа соответствует некоторый элемент электрической цепи. Составить линейно независимые системы уравнений Кирхгофа для токов и напряжений. Пусть первому и пятому ребру соответствуют источники тока с ЭДС E1 и E2, а остальные элементы являются сопротивлениями. Используя закон Ома, и, предполагая внутренние сопротивления источников тока равными нулю, получить систему уравнений для токов.

%\vspace{\baselineskip}

\includeimage{0.8}{images/6_0.pdf}

{\large\textbf{Решение:}}

\begin{enumerate}
  \item Зададим на графе произвольную ориентацию:

  \includeimage{0.8}{images/6_1.pdf}

  \item Построим произвольное остовное дерево D заданного графа:

  \includeimage{0.8}{images/6_2.pdf}

  \newpage

  \item Найдем базис циклов, добавляя к остовному дереву по одному не вошедшему в него ребру. Затем найдем соответствующие вектор-циклы.

  $(D+q_1):\mu_1:V_2-V_1-V_4-V_3-V_2\Longrightarrow C(\mu_1)=(1,1,1,1,0,0,0,0,0,0,0,0)$

  $(D+q_5):\mu_2:V_8-V_1-V_4-V_3-V_2-V_7-V_6-V_5-V_8\Longrightarrow C(\mu_2)=(0,1,1,1,1,1,0,0,0,1,1,1)$

  $(D+q_7):\mu_3:V_6-V_3-V_2-V_7-V_6\Longrightarrow C(\mu_3)=(0,1,0,0,0,1,1,0,0,1,0,0)$

  $(D+q_8):\mu_4:V_5-V_4-V_3-V_2-V_7-V_6-V_5\Longrightarrow C(\mu_4)=(0,1,0,1,0,1,0,1,0,1,1,0)$

  $(D+q_9):\mu_5:V_8-V_7-V_6-V_5-V_8\Longrightarrow C(\mu_5)=(0,0,0,0,0,0,0,0,1,1,1,1)$

  \item Цикломатическая матрица графа имеет вид:

  $$C=\left( \begin{array}{*{12}{c}}1 & 1 & 1 & 1 & 0 & 0 & 0 & 0 & 0 & 0 & 0 & 0\\0 & 1 & 1 & 1 & 1 & 1 & 0 & 0 & 0 & 1 & 1 & 1\\0 & 1 & 0 & 0 & 0 & 1 & 1 & 0 & 0 & 1 & 0 & 0\\0 & 1 & 0 & 1 & 0 & 1 & 0 & 1 & 0 & 1 & 1 & 0\\0 & 0 & 0 & 0 & 0 & 0 & 0 & 0 & 1 & 1 & 1 & 1 \end{array} \right)$$

  \item Выпишем закон Кирхгова для напряжений:

  $$\left( \begin{array}{*{12}{c}}1 & 1 & 1 & 1 & 0 & 0 & 0 & 0 & 0 & 0 & 0 & 0\\0 & 1 & 1 & 1 & 1 & 1 & 0 & 0 & 0 & 1 & 1 & 1\\0 & 1 & 0 & 0 & 0 & 1 & 1 & 0 & 0 & 1 & 0 & 0\\0 & 1 & 0 & 1 & 0 & 1 & 0 & 1 & 0 & 1 & 1 & 0\\0 & 0 & 0 & 0 & 0 & 0 & 0 & 0 & 1 & 1 & 1 & 1 \end{array} \right)*\begin{pmatrix} U_1 \\ U_2 \\ U_3 \\ U_4 \\ U_5 \\ U_6 \\ U_7 \\ U_8 \\ U_9 \\ U_{10} \\ U_{11} \\ U_{12} \end{pmatrix}=0$$

  Напряжения, соответствующие ребрам, не вошедшим в остовное дерево -- базисные переменные системы.

  $$ \left\{ \begin{array}{l}U_1+U_2+U_3+U_4=0 \\U_2+U_3+U_4+U_5+U_6+U_{10}+U_{11}+U_{12}=0 \\U_2+U_6+U_7+U_{10}=0\\U_2+U_4+U_6+U_8+U_{10}+U_{11}=0 \\U_9+U_{10}+U_{11}+U_{12}=0\end{array} \right.$$

   $$\left\{ \begin{array}{l}U_1=-U_2-U_3-U_4 \\U_5=-U_2-U_3-U_4-U_6-U_{10}-U_{11}-U_{12} \\U_7=-U_2-U_6-U_{10}\\U_8=-U_2-U_4-U_6-U_{10}-U_{11}\\U_9=-U_{10}-U_{11}-U_{12}\end{array}\right.$$

   \item Выпишем закон Кирхгова для токов:

   \newpage

   \item Выпишем уравнения Кирхгофа для токов.

   Найдем матрицу инцидентности орграфа:

   \begin{tabular}{|*{13}{c|}}
     \hline
      & $q_1$ & $q_2$ & $q_3$ & $q_4$ & $q_5$ & $q_6$ & $q_7$ & $q_8$ & $q_9$ & $q_{10}$ & $q_{11}$ & $q_{12}$ \\
      \hline
      $V_1$ & 1 & 0 & -1 & 0 & 1 & 0 & 0 & 0 & 0 & 0 & 0 & 0 \\
      \hline
      $V_2$ & -1 & 1 & 0 & 0 & 0 & -1 & 0 & 0 & 0 & 0 & 0 & 0 \\
      \hline
      $V_3$ & 0 & -1 & 0 & 1 & 0 & 0 & 1 & 0 & 0 & 0 & 0 & 0 \\
      \hline
      $V_4$ & 0 & 0 & 1 & -1 & 0 & 0 & 0 & 1 & 0 & 0 & 0 & 0 \\
      \hline
      $V_5$ & 0 & 0 & 0 & 0 & 0 & 0 & 0 & -1 & 0 & 0 & 1 & -1 \\
      \hline
      $V_6$ & 0 & 0 & 0 & 0 & 0 & 0 & -1 & 0 & 0 & 1 & -1 & 0 \\
      \hline
      $V_7$ & 0 & 0 & 0 & 0 & 0 & 1 & 0 & 0 & 1 & -1 & 0 & 0 \\
      \hline
      $V_8$ & 0 & 0 & 0 & 0 & -1 & 0 & 0 & 0 & -1 & 0 & 0 & 1 \\
      \hline
   \end{tabular}

   \vspace{\baselineskip}

   $$B=\left( \begin{array}{*{12}{c}}
   1 & 0 & -1 & 0 & 1 & 0 & 0 & 0 & 0 & 0 & 0 & 0 \\
   -1 & 1 & 0 & 0 & 0 & -1 & 0 & 0 & 0 & 0 & 0 & 0 \\
   0 & -1 & 0 & 1 & 0 & 0 & 1 & 0 & 0 & 0 & 0 & 0 \\
   0 & 0 & 1 & -1 & 0 & 0 & 0 & 1 & 0 & 0 & 0 & 0 \\
   0 & 0 & 0 & 0 & 0 & 0 & 0 & -1 & 0 & 0 & 1 & -1 \\
   0 & 0 & 0 & 0 & 0 & 0 & -1 & 0 & 0 & 1 & -1 & 0 \\
   0 & 0 & 0 & 0 & 0 & 1 & 0 & 0 & 1 & -1 & 0 & 0 \\
   0 & 0 & 0 & 0 & -1 & 0 & 0 & 0 & -1 & 0 & 0 & 1 \\
   \end{array} \right)$$

   $$\left( \begin{array}{*{12}{c}}
   1 & 0 & -1 & 0 & 1 & 0 & 0 & 0 & 0 & 0 & 0 & 0 \\
   -1 & 1 & 0 & 0 & 0 & -1 & 0 & 0 & 0 & 0 & 0 & 0 \\
   0 & -1 & 0 & 1 & 0 & 0 & 1 & 0 & 0 & 0 & 0 & 0 \\
   0 & 0 & 1 & -1 & 0 & 0 & 0 & 1 & 0 & 0 & 0 & 0 \\
   0 & 0 & 0 & 0 & 0 & 0 & 0 & -1 & 0 & 0 & 1 & -1 \\
   0 & 0 & 0 & 0 & 0 & 0 & -1 & 0 & 0 & 1 & -1 & 0 \\
   0 & 0 & 0 & 0 & 0 & 1 & 0 & 0 & 1 & -1 & 0 & 0 \\
   0 & 0 & 0 & 0 & -1 & 0 & 0 & 0 & -1 & 0 & 0 & 1 \\
   \end{array} \right)*\begin{pmatrix}I_1 \\ I_2 \\ I_3 \\ I_4 \\ I_5 \\ I_6 \\ I_7 \\ I_8 \\ I_9 \\ I_{10} \\ I_{11} \\ I_{12}\end{pmatrix} = 0$$

   $\left\{
      \begin{array}{l}
        I_1-I_3+I_5=0\\
        I_2-I_1-1_6=0\\
        I_4-I_2+I_7=0\\
        I_3-I_4+I_8=0\\
        I_{11}-I_{12}-I_8=0 \text{ ЛЗ}\\
        I_{10}-I_7-I_{11}=0\\
        I_6+I_9-I_{10}=0 \\
        I_{12}-I_5-I_9=0 \\
      \end{array}
    \right. \Longrightarrow
    \left\{
      \begin{array}{l}
        I_1-I_3+I_5=0\\
        I_2-I_1-1_6=0\\
        I_4-I_2+I_7=0\\
        I_3-I_4+I_8=0\\
        I_{10}-I_7-I_{11}=0\\
        I_6+I_9-I_{10}=0 \\
        I_{12}-I_5-I_9=0 \\
      \end{array}
    \right.$

  \item Подставим Закон Ома

  $\left\{
    \begin{array}{l}
      E_1=-I_2R_2-I_3R_3-I_4R_4\\
      E_2=-I_2R_2-I_3R_3-I_4R_4-I_6R_6-I_{10}R_{10}-I{11}R_{11}-I_{12}R_{12}\\
      0=I_2R_2+I_6R_6+I_7R_7+I_{10}R_{10}\\
      0=I_2R_2+I_4R_4+I_6R_6+I_8R_8+I_{10}R_{10}+I_{11}R_{11}\\
      0=I_9R_9+I_{10}R_{10}+I_{11}R_{11}+I_{12}R_{12}\\
    \end{array}
  \right.$

\newpage

\item Совместная система имеет вид:

$\left\{
  \begin{array}{l}
    I_1-I_3+I_5=0\\
    I_2-I_1-1_6=0\\
    I_4-I_2+I_7=0\\
    I_3-I_4+I_8=0\\
    I_{10}-I_7-I_{11}=0\\
    I_6+I_9-I_{10}=0 \\
    I_{12}-I_5-I_9=0 \\
    E_1=-I_2R_2-I_3R_3-I_4R_4\\
    E_2=-I_2R_2-I_3R_3-I_4R_4-I_6R_6-I_{10}R_{10}-I{11}R_{11}-I_{12}R_{12}\\
    0=I_2R_2+I_6R_6+I_7R_7+I_{10}R_{10}\\
    0=I_2R_2+I_4R_4+I_6R_6+I_8R_8+I_{10}R_{10}+I_{11}R_{11}\\
    0=I_9R_9+I_{10}R_{10}+I_{11}R_{11}+I_{12}R_{12}\\
  \end{array}
\right.$

\end{enumerate}

\newpage
