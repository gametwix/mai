\section*{Задание 1}
Определить для орграфа, заданного матрицей смежности:
\begin{enumerate}
\item[a)]Матрицу односторонней связности;
\item[б)]Матрицу сильной связности;
\item[в)]Компоненты сильной связности;
\item[г)]Матрицу контуров.
\end{enumerate}

$A = \begin{pmatrix}0 & 0 & 0 & 1 \\1 & 0 & 1 & 1\\0 & 0 & 0 & 1\\0 & 0 & 1 & 0\end{pmatrix}$

\vspace{\baselineskip}

{\large\textbf{Решение:}}

\vspace{\baselineskip}

а) Найдем матрицу односторонней связности по формуле: $T=E\vee A \vee A^2 \vee A^3$.

\begin{enumerate}
\item[1)]$A^2 = \begin{pmatrix}0 & 0 & 0 & 1 \\1 & 0 & 1 & 1\\0 & 0 & 0 & 1\\0 & 0 & 1 & 0\end{pmatrix}*\begin{pmatrix}0 & 0 & 0 & 1 \\1 & 0 & 1 & 1\\0 & 0 & 0 & 1\\0 & 0 & 1 & 0\end{pmatrix} = \begin{pmatrix}0 & 0 & 1 & 0 \\0 & 0 & 1 & 1\\0 & 0 & 1 & 0\\0 & 0 & 0 & 1\end{pmatrix}$

\item[2)]$A^3 = \begin{pmatrix}0 & 0 & 1 & 0 \\0 & 0 & 1 & 1\\0 & 0 & 1 & 0\\0 & 0 & 0 & 1\end{pmatrix}*\begin{pmatrix}0 & 0 & 0 & 1 \\1 & 0 & 1 & 1\\0 & 0 & 0 & 1\\0 & 0 & 1 & 0\end{pmatrix} = \begin{pmatrix}0 & 0 & 0 & 1 \\0 & 0 & 1 & 1\\0 & 0 & 0 & 1\\0 & 0 & 1 & 0\end{pmatrix}$

\item[3)]$T=E\vee A \vee A^2 \vee A^3=\begin{pmatrix}1 & 0 & 0 & 0 \\0 & 1 & 0 & 0\\0 & 0 & 1 & 0\\0 & 0 & 0 & 1\end{pmatrix} \vee \begin{pmatrix}0 & 0 & 0 & 1 \\1 & 0 & 1 & 1\\0 & 0 & 0 & 1\\0 & 0 & 1 & 0\end{pmatrix} \vee \begin{pmatrix}0 & 0 & 1 & 0 \\0 & 0 & 1 & 1\\0 & 0 & 1 & 0\\0 & 0 & 0 & 1\end{pmatrix} \vee \begin{pmatrix}0 & 0 & 0 & 1 \\0 & 0 & 1 & 1\\0 & 0 & 0 & 1\\0 & 0 & 1 & 0\end{pmatrix} = \newline=\begin{pmatrix}1 & 0 & 1 & 1 \\1 & 1 & 1 & 1\\0 & 0 & 1 & 1\\0 & 0 & 1 & 1\end{pmatrix} = T \text{ -- матрица односторонней связности}$ 
\end{enumerate}

б) Матрица сильной связности: $\overline{S} = T\&T^{T}$

\vspace{\baselineskip}

$\overline{S} = T\&T^{T} = \begin{pmatrix}1 & 0 & 1 & 1 \\1 & 1 & 1 & 1\\0 & 0 & 1 & 1\\0 & 0 & 1 & 1\end{pmatrix} \& \begin{pmatrix}1 & 1 & 0 & 0 \\0 & 1 & 0 & 0\\1 & 1 & 1 & 1\\1 & 1 & 1 & 1\end{pmatrix} = \begin{pmatrix}1 & 0 & 0 & 0 \\0 & 1 & 0 & 0\\0 & 0 & 1 & 1\\0 & 0 & 1 & 1\end{pmatrix}$

\vspace{\baselineskip}

в) Компоненты сильной связности

Выбираем первую строку, как ненулевую в матрице сильной связности

\vspace{\baselineskip}

$\overline{S} = \begin{pmatrix}1 & 0 & 0 & 0 \\0 & 1 & 0 & 0\\0 & 0 & 1 & 1\\0 & 0 & 1 & 1\end{pmatrix}$

\vspace{\baselineskip}

Номера вершин первой компоненты сильной связности соответствуют номерам столбцов матрицы $\overline{S}$, в которых в первой строке стоят единицы: $\{\mathcal{V}_1\}$.

\begin{enumerate}
\item Обнуляем первый столбец матрицы $\overline{S}$. Получаем матрицу

$\overline{S_1} = \begin{pmatrix}0 & 0 & 0 & 0 \\0 & 1 & 0 & 0\\0 & 0 & 1 & 1\\0 & 0 & 1 & 1\end{pmatrix}$

\item Ищем ненулевую строку матрицы ${S_1}$: это вторая строка. Единица одна -- во втором столбце. Следовательно, вторая компонента сильной связности: $\{\mathcal{V}_2\}$.

\item Обнуляем первый столбец матрицы $\overline{S_1}$. Получаем матрицу

$\overline{S_2} = \begin{pmatrix}0 & 0 & 0 & 0 \\0 & 0 & 0 & 0\\0 & 0 & 1 & 1\\0 & 0 & 1 & 1\end{pmatrix}$

\item Ищем ненулевую строку матрицы ${S_2}$: это третья строка. Единицы две -- в третьем и четвертом столбце. Следовательно, третья компонента сильной связности: $\{\mathcal{V}_3 , \mathcal{V}_4\}$.
\end{enumerate}

г) Матрица контуров: $K = \overline{S}\&A$.

$$K = \begin{pmatrix}1 & 0 & 0 & 0 \\0 & 1 & 0 & 0\\0 & 0 & 1 & 1\\0 & 0 & 1 & 1\end{pmatrix}\&\begin{pmatrix}0 & 0 & 0 & 1 \\1 & 0 & 1 & 1\\0 & 0 & 0 & 1\\0 & 0 & 1 & 0\end{pmatrix} = \begin{pmatrix}0 & 0 & 0 & 0 \\0 & 0 & 0 & 0\\0 & 0 & 0 & 1\\0 & 0 & 1 & 0\end{pmatrix}$$

Cлудовательно, дуги $<\mathcal{V}_3,\mathcal{V}_4>$,$<\mathcal{V}_4,\mathcal{V}_3>$ принадлежат контуру графа.

\newpage